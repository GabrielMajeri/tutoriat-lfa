\documentclass[a4paper, 12pt]{article}

\usepackage{fontenc}

% Suport pentru limba română
\usepackage{polyglossia}
\setdefaultlanguage{romanian}

% Schimbă textul pentru demonstrație
\addto\captionsromanian{\renewcommand*{\proofname}{Rezolvare}}

% Notații matematice
\usepackage{mathtools}
\usepackage{amsthm}
\usepackage{unicode-math}

\AtBeginDocument{
  \renewcommand{\setminus}{\mathbin{\backslash}}%
}

\theoremstyle{definition}
\newtheorem{exercise}{Exercițiul}[subsection]
\renewcommand{\theexercise}{\arabic{exercise}}

\theoremstyle{remark}
\newtheorem*{observation}{Observație}

% Pentru notația cu mulțimi
\usepackage{braket}

% Pentru a muta mai sus titlul
\usepackage{titling}
\setlength{\droptitle}{-5em}

% Link-uri funcționale în text
\usepackage{hyperref}
\hypersetup{
    colorlinks=true,
    urlcolor=blue,
    linkcolor=cyan,
}

% Pentru a desena diagrame și grafuri
\usepackage{tikz}
\usetikzlibrary{automata, positioning, arrows}
\tikzset{
    ->, % makes the edges directed
    >=stealth, % makes the arrow heads bold
    node distance=3cm, % specifies the minimum distance between two nodes. Change if necessary.
    every state/.style={thick, fill=gray!10}, % sets the properties for each ’state’ node
    initial text=$ $, % sets the text that appears on the start arrow
}

% Etichete customizabile
\usepackage{caption}

% Figuri care sunt fixate
\usepackage{float}

\title{Limbaje Formale și Automate\\Tutoriat}
\author{Gabriel Majeri}
\date{}

% Comenzi utile pentru notația matematică

\newcommand{\naturals}{\symbb{N}}

\newcommand{\lang}{\symcal{L}}

\newcommand{\dfa}{DFA}
\newcommand{\nfa}{NFA}
\newcommand{\lnfa}{\(\lambda\)-NFA}

\newcommand{\reglang}{\symrm{REG}}
\newcommand{\cfglang}{\symrm{CFG}}

\newcommand{\pda}{PDA}
\newcommand{\dpda}{DPDA}
\newcommand{\npda}{NPDA}

\newcommand{\abs}[1]{\left\lvert#1\right\rvert}

\newcommand{\bigO}{\symcal{O}}

\begin{document}

\maketitle

\section*{Introducere}

Cursul este destul de teoretic, iar la examen veți avea atât o parte de teorie cât și o parte de exerciții. Pentru teorie ar fi bine să știți \textbf{algebră} și câteva noțiuni legate de \textbf{grafuri} pentru că demonstrațiile conțin noțiuni din aceste materii.

Laboratorul explorează aplicațiile \textbf{practice} ale noțiunilor discutate la curs, cum ar fi utilizarea \href{https://en.wikipedia.org/wiki/Regular_expression}{expresiilor regulate} pentru \href{https://docs.python.org/3/library/re.html}{căutarea în text}, sau importanța \href{https://en.wikipedia.org/wiki/Formal_grammar}{gramaticilor} în \href{https://en.wikipedia.org/wiki/LALR_parser}{compilatoare}.

\newpage

\section*{Notare}

Acestea sunt câteva sfaturi practice legate de notarea la această materie (relevante în special pentru seria 13).

\subsection*{Examen}

Timpul alocat pentru rezolvarea subiectelor este \textbf{2 ore}. Modele de examen din anii trecuți se găsesc pe \href{https://github.com/palcu/fmi/blob/master/lfa.md}{palcu/fmi}.

\subsubsection*{Teorie}
Pe partea de teorie, întrebările sunt de forma:
\begin{itemize}
    \item Să demonstrezi o teoremă sau o propoziție din curs;

          Exemple:
          \begin{itemize}
              \item Demonstrați lema de pompare pentru limbaje independente de context;
              \item Demonstrați prima implicație a teoremei Myhill-Nerode;
              \item Demonstrați trei proprietăți ale limbajelor regulate.
          \end{itemize}

    \item Să justifici dacă o anumită afirmație este sau nu adevărată;

          Exemple:
          \begin{itemize}
              \item Limbajele regulate sunt închise la complementare;
              \item Fie limbajele \(\limbaj_1, \limbaj_2\) cu \(\limbaj_1 \subset \limbaj_2\) și \(\limbaj_2 \in \reglang\). Atunci \(\limbaj_1 \in \reglang\).
          \end{itemize}

    \item Să justifici dacă o anumită problemă este sau nu decidabilă.

          Exemple:
          \begin{itemize}
              \item Argumentați dacă este decidabilă egalitatea între două limbaje regulate;
              \item Justificați dacă determinarea intersecției a două limbaje independente de context este decidabilă.
          \end{itemize}
\end{itemize}

\subsubsection*{Exerciții}

Exemple de exerciții pe care le puteți primii la examen:
\begin{itemize}
    \item Se dă un limbaj, dacă acesta este regulat/independent de context scrieți un automat sau o expresie regulată/o gramatică care să îl accepte.
          Dacă nu, demonstrați că nu este regulat/nu este independent de context (folosind lema de pompare respectivă).

    \item Se dau două \dfa/\nfa/\lnfa: să se minimizeze, să se verifice dacă acceptă același limbaj, sau să se calculeze intersecția/reuniunea lor.

    \item Să se dea exemplu de o gramatică independentă de context care respectă anumite condiții: să aibă un anumit număr de terminali/neterminali, un anumit număr de producții, să genereze un limbaj finit, etc.

    \item Se dă o gramatică independentă de context, să se aducă la o formă normală Chomsky, sau să se elimine producțiile unitate, etc.

    \item Să se scrie un automat push-down determinist/nedeterminist care să accepte un limbaj prin stare vidă/prin stare finală/prin stare finală și stivă vidă.
\end{itemize}

\subsection*{Laborator}

Îți alegi teme (exerciții) pe care le rezolvi individual și apoi le prezinți la laborator. În general, temele mai grele (care necesită mai mult efort) au note maxime mai mari (dar și penalitate mai mare dacă nu le faci).

\section*{Limbaje formale}

Orice problemă de decizie (adică cu răspuns „da” sau „nu”) din informatică poate fi redusă la determinarea apartenenței la un limbaj.
De exemplu, verificarea că un număr \(k\) este \emph{prim}, se poate face verificând dacă cuvântul \(\underbrace{aaa \dots aaa}_{k \text{ ori}}\) aparține limbajului \(\Set{ a^p | p \text{ prim} }\).

\subsection*{Limbaje regulate}

\textbf{Cum le recunoaștem:} Principala limitare a \dfa/\nfa/\lnfa{} este că nu au memorie. În formula pentru limbajele regulate pot apărea doar condiții liniare, de forma \(a^{nk + m}\), și dacă apar mai mulți indici la putere, aceștia nu sunt corelați.
s
\begin{figure}[H]
    \centering
    \begin{tikzpicture}
        \node[state, initial] (1) {\(q_1\)};
        \node[state, accepting, right of=1] (2) {\(q_2\)};

        \draw (1) edge[bend left, above] node{a} (2)
        (2) edge[bend left, below] node{a} (1);
    \end{tikzpicture}
    \caption*{Un DFA care acceptă limbajul \(\limbaj = \Set{a^{2k + 1} | k \in \naturals}\)}
\end{figure}

\begin{observation}
    Orice limbaj finit este regulat. Putem construi un DFA care să aibă câte o stare finală pentru fiecare cuvânt. Astfel se obține un \href{https://en.wikipedia.org/wiki/Trie}{trie}.
\end{observation}

\textbf{Proprietăți:} închise la intersecție, reuniune, complement, diferență de mulțime. Sunt închise și la concatenare și la stelare, și la morfisme și morfisme inverse.

\subsection*{Limbaje independente de context}

\textbf{Cum le recunoaștem:} Putem avea indici corelați (de exemplu \(a^{2k} b^{3k}\)).

Dacă apar mai mulți indici corelați, aceștia ar trebui să se grupeze asemenea parantezelor corect închise. De exemplu, se poate arăta că \(a^n b^m c^m d^n\) este independent de context dar \(a^n b^m c^n d^m\) \textbf{nu} este.

De asemenea, nu pot fi mai mult de două variabile corelate. De exemplu \(a^n b^n c^n\) este exemplul clasic de limbaj care \textbf{nu} este independent de context.

\textbf{Proprietăți:} închise la toate operațiile menționate mai sus \textbf{cu excepția} intersecție, complement, sau diferență. Sunt închise totuși la intersecția \emph{cu un limbaj regulat} (acest lucru ne ajută în exerciții).

\subsection*{Exerciții}

\begin{exercise}
    Scrieți un automat finit determinist care să accepte limbajul
    \[\limbaj = \Set{ a^{2k} | k \geq 0 }\]
\end{exercise}
\begin{proof}
    Cel mai simplu automat care acceptă acest limbaj este:
    \begin{figure}[H]
        \centering
        \begin{tikzpicture}
            \node[state, accepting, initial] (1) {\(q_1\)};
            \node[state, right of=1] (2) {\(q_2\)};

            \draw (1) edge[bend left, above] node{a} (2)
            (2) edge[bend left, below] node{a} (1);
        \end{tikzpicture}
    \end{figure}
\end{proof}

\begin{exercise}[exercițiul 7 din examen iunie 2011]
    Dați exemplu de un \nfa care nu este nici \lnfa, nici \dfa. Automatul trebuie să aibe minim 6 stări accesibile din starea inițială. Transformați automatul într-un \dfa.
\end{exercise}
\begin{proof}
    Ni se cere ca stările să fie accesibile, nu neapărat utile. Putem să modificăm automatul de mai sus:
    \begin{figure}[H]
        \centering
        \begin{tikzpicture}
            \node[state, accepting, initial] (1) {\(q_1\)};
            \node[state, right of=1] (2) {\(q_2\)};
            \node[state, above right of=1] (3) {\(q_3\)};
            \node[state, above left of=1] (4) {\(q_4\)};
            \node[state, below right of=1] (5) {\(q_5\)};
            \node[state, below left of=1] (6) {\(q_6\)};

            \draw (1) edge[bend left, above] node{a} (2)
            (2) edge[bend left, below] node{a} (1)
            (1) edge[bend left, above] node{a} (3)
            (1) edge[bend right, right] node{a} (4)
            (1) edge[bend right, below] node{a} (5)
            (1) edge[bend left, right] node{a} (6);
        \end{tikzpicture}
    \end{figure}

    Folosim metoda tabelului pentru a obține \dfa-ul echivalent:
    \begin{center}
        \begin{tabular}{c|c}
                                                   & \(a\)                     \\
            \hline
            Notăm cu \(A = \Set{q_1}\)             & \(\Set{q_2, \dots, q_6}\) \\
            \hline
            Notăm cu \(B = \Set{q_2, \dots, q_6}\) & \(\Set{q_1}\)
        \end{tabular}
    \end{center}
    Obținem DFA-ul inițial, unde \(q_1\) este \(A\) și \(q_2\) este \(B\).
\end{proof}

\begin{exercise}[bazat pe exercițiul 9 din examen iunie 2011]
    Construiți o gramatică independentă de context care să genereze limbajul
    \[\limbaj = \Set{ a^{2k} b^l a^k | k \geq 0, l \geq 1 }\]
\end{exercise}
\begin{proof}
    Vom genera partea exterioară a cuvântului (\textit{a}-urile) prin \(A\) și oricâte \textit{b}-uri prin \(B\).
    \begin{align*}
        S & \rightarrow A    \\
        A & \rightarrow aaAa \\
        A & \rightarrow B    \\
        B & \rightarrow bB   \\
        B & \rightarrow b
    \end{align*}
\end{proof}

\begin{exercise}[exercițiul 8 din examen iunie 2013]
    Construiți o gramatică independentă de context care să genereze limbajul
    \[\limbaj = \Set{ 0^{4k} 1^l 0^k | k \geq 0, l \geq 1 } \cdot \Set{ 0^i 1^{j + 3} | i \neq j}\]
\end{exercise}
\begin{proof}
    Pentru a genera prima mulțime, putem folosi producțiile:
    \begin{align*}
        S_1 & \rightarrow 0^4 S 0 \mid A \\
        A   & \rightarrow 1 A \mid 1
    \end{align*}
    Pentru a genera un număr diferit de \(0\) și de \(1\), trebuie să generăm unul din două cazuri: \(i > j\) sau \(i < j\).
    \begin{align*}
        S_2 & \rightarrow U \mid V                     \\
        U   & \rightarrow 0 U \mid 0 T \tag{\(i > j\)} \\
        V   & \rightarrow 1 V \mid 1 T \tag{\(i < j\)} \\
        T   & \rightarrow 0 T 1 \mid \lambda
    \end{align*}
    Le putem unii prin producția:
    \[S \rightarrow S_1 S_2 1^3\]
\end{proof}

\begin{exercise}[exercițiul 10 din examen iunie 2018]
    Construiți o gramatică independentă de context care să genereze limbajul
    \[\limbaj = \Set{ a^{m + n} b^k a^{m + k + i} b^n | i, k, m, n \geq 1 }\]
\end{exercise}
\begin{proof}
    Putem rescrie limbajul ca:
    \[\limbaj = \Set{ a^n a^m b^k a^k a^m a^i b^n | i, k, m, n \geq 1 }\]
    Acum se vede mult mai ușor că se poate genera limbajul:
    \begin{align*}
        S & \rightarrow a S b \mid a A b \tag{generează \(a^n \dots b^n\)} \\
        A & \rightarrow B C                                                \\
        B & \rightarrow a B a \mid a D a \tag{generează \(a^m \dots a^m\)} \\
        C & \rightarrow aC \mid a \tag{generează \(a^i\)}                  \\
        D & \rightarrow bDa \mid ba \tag{generează \(b^k a^k\)}
    \end{align*}
\end{proof}

\begin{exercise}[exercițiul 7, subpunctul \textit{a} din examen iunie 2016]
    Dați o gramatică independentă de context cu 7 producții, 2 din ele să fie producții unitare (unit production), și care are cel puțin 2 simboluri neterminale (nonterminating) și un simbol inaccesbil (unreachable).
\end{exercise}
\begin{proof}
    \begin{align*}
        S \rightarrow AB \mid C \mid \lambda                                  \\
        A \rightarrow a, B \rightarrow b \tag{\(A, B\) simboluri neterminale} \\
        C \rightarrow A, C \rightarrow B \tag{producții unitare}              \\
        X \rightarrow Y \tag{două simboluri inaccesibile}
    \end{align*}
\end{proof}

\begin{exercise}[exercițiul 7 din examen iunie 2017]
    Pentru următorul automat dați fiecare pas din algoritmul de construire al expresiei regulate echivalente:
    \begin{figure}[H]
        \centering
        \begin{tikzpicture}
            \node[state, initial] (1) {\(q_1\)};
            \node[state, accepting, right of=1] (2) {\(q_2\)};
            \node[state, right of=2] (3) {\(q_3\)};
            \node[state, accepting, below right of=2] (4) {\(q_4\)};

            \draw (1) edge[loop above] node{a} (1)
            (1) edge[above] node{b} (2)
            (2) edge[loop above] node{b} (2)
            (2) edge[bend left, above] node{a} (3)
            (2) edge[bend right, below] node{a} (4)
            (3) edge[loop right] node{c} (3)
            (3) edge[bend left, right] node{b} (4)
            (4) edge[bend right, above] node{a} (2);
        \end{tikzpicture}
    \end{figure}
\end{exercise}
\begin{proof}
    Realizăm o serie de transformări plecând de la automatul inițial.
    \begin{figure}[H]
        \centering
        \caption*{Adăugăm o nouă stare inițială și modificăm automatul să aibă o singură stare finală:}
        \begin{tikzpicture}
            \node[state, initial] (0) {\(q_0\)};
            \node[state, right of=0] (1) {\(q_1\)};
            \node[state, right of=1] (2) {\(q_2\)};
            \node[state, right of=2] (3) {\(q_3\)};
            \node[state, below right of=2] (4) {\(q_4\)};
            \node[state, accepting, below of=2] (5) {\(q_5\)};

            \draw (0) edge[above] node{\(\lambda\)} (1)
            (1) edge[loop above] node{a} (1)
            (1) edge[above] node{b} (2)
            (2) edge[loop above] node{b} (2)
            (2) edge[bend left, above] node{a} (3)
            (2) edge[bend right, below] node{a} (4)
            (2) edge[bend right, left] node{\(\lambda\)} (5)
            (3) edge[loop right] node{c} (3)
            (3) edge[bend left, right] node{b} (4)
            (4) edge[bend right, above] node{a} (2)
            (4) edge[bend left, below] node{\(\lambda\)} (5);
        \end{tikzpicture}
    \end{figure}
    \begin{figure}[H]
        \centering
        \caption*{Începem să eliminăm din stări. Începem cu \(q_1\):}
        \begin{tikzpicture}
            \node[state, initial] (0) {\(q_0\)};
            \node[state, right of=0] (2) {\(q_2\)};
            \node[state, right of=2] (3) {\(q_3\)};
            \node[state, below right of=2] (4) {\(q_4\)};
            \node[state, accepting, below of=2] (5) {\(q_5\)};

            \draw (0) edge[above] node{\(a^* b\)} (2)
            (2) edge[loop above] node{b} (2)
            (2) edge[bend left, above] node{a} (3)
            (2) edge[bend right, below] node{a} (4)
            (2) edge[bend right, left] node{\(\lambda\)} (5)
            (3) edge[loop right] node{c} (3)
            (3) edge[bend left, right] node{b} (4)
            (4) edge[bend right, above] node{a} (2)
            (4) edge[bend left, below] node{\(\lambda\)} (5);
        \end{tikzpicture}
    \end{figure}
    \begin{figure}[H]
        \centering
        \caption*{Eliminăm \(q_3\):}
        \begin{tikzpicture}
            \node[state, initial] (0) {\(q_0\)};
            \node[state, right of=0] (2) {\(q_2\)};
            \node[state, below right of=2] (4) {\(q_4\)};
            \node[state, accepting, below of=2] (5) {\(q_5\)};

            \draw (0) edge[above] node{\(a^* b\)} (2)
            (2) edge[loop above] node{b} (2)
            (2) edge[bend left=75, right] node{\(a c^* b\)} (4)
            (2) edge[bend right, below] node{a} (4)
            (2) edge[bend right, left] node{\(\lambda\)} (5)
            (4) edge[bend right, above] node{a} (2)
            (4) edge[bend left, below] node{\(\lambda\)} (5);
        \end{tikzpicture}
    \end{figure}
    \begin{figure}[H]
        \centering
        \caption*{Reunim muchiile de la \(q_2\) la \(q_4\):}
        \begin{tikzpicture}
            \node[state, initial] (0) {\(q_0\)};
            \node[state, right of=0] (2) {\(q_2\)};
            \node[state, right of=2] (4) {\(q_4\)};
            \node[state, accepting, below right of=2] (5) {\(q_5\)};

            \draw (0) edge[above] node{\(a^* b\)} (2)
            (2) edge[loop above] node{b} (2)
            (2) edge[bend left, above] node{\(a + a c^* b\)} (4)
            (2) edge[bend right, left] node{\(\lambda\)} (5)
            (4) edge[bend left, below] node{a} (2)
            (4) edge[bend left, right] node{\(\lambda\)} (5);
        \end{tikzpicture}
    \end{figure}
    \begin{figure}[H]
        \centering
        \caption*{Eliminăm starea \(q_4\):}
        \begin{tikzpicture}
            \node[state, initial] (0) {\(q_0\)};
            \node[state, right of=0] (2) {\(q_2\)};
            \node[state, accepting, below right of=2] (5) {\(q_5\)};

            \draw (0) edge[above] node{\(a^* b\)} (2)
            (2) edge[loop above] node{b} (2)
            (2) edge[bend left, right] node{\((((a + a c^* b) a)^* b^*)^* \cdot (a + ac^* b)\)} (5)
            (2) edge[bend right, left] node{\(\lambda\)} (5);
        \end{tikzpicture}
    \end{figure}
    \begin{figure}[H]
        \centering
        \caption*{Reunim muchiile de la \(q_2\) la \(q_5\):}
        \begin{tikzpicture}
            \node[state, initial] (0) {\(q_0\)};
            \node[state, right of=0] (2) {\(q_2\)};
            \node[state, accepting, below right of=2] (5) {\(q_5\)};

            \draw (0) edge[above] node{\(a^* b\)} (2)
            (2) edge[loop above] node{b} (2)
            (2) edge[bend left, right] node{\(\lambda + (((a + a c^* b) a)^* b^*)^* \cdot (a + ac^* b)\)} (5);
        \end{tikzpicture}
    \end{figure}

    \begin{figure}[H]
        \centering
        \caption*{Eliminăm starea \(q_2\):}
        \begin{tikzpicture}
            \node[state, initial] (0) {\(q_0\)};
            \node[state, accepting, right of=0] (5) {\(q_5\)};

            \draw (0) edge[below] node{\(E\)} (5);
        \end{tikzpicture}
    \end{figure}

    Unde \(E\) este expresia regulată cerută:
    \[a^* b b^* \cdot (\lambda + (((a + a c^* b) a)^* b^*)^* \cdot (a + ac^* b))\]
\end{proof}

\section*{Lema de pompare}

Să se demonstreze că limbajul
\[\limbaj = \Set{ a^n b^n c^2 | n \in \naturals^* }\]
nu este regulat.

Pentru a demonstra că acesta nu este un limbaj regulat, trebuie să demonstrăm că nu există niciun \dfa/\nfa/\lnfa{} care să îl accepte.

Când vrei să demonstrezi că ceva nu este posibil în matematică, de obicei folosești \emph{reducerea la absurd}. Presupui că o propoziție este adevărată, și ajungi la o contradicție.

O demonstrație că un limbaj nu este regulat/nu este independent de context decurge în felul următor:
\begin{enumerate}
    \item Presupun că limbajul este regulat/independent de context.

    \item \textbf{Dacă mă ajută,} pot să aplic orice operații la care este închis (deoarece am presupus că limbajul inițial este regulat/independent de context, în urma acestor operații obțin tot un limbaj regulat/independent de context):
          \begin{itemize}
              \item pentru limbaje regulate: cam toate operațiile
              \item pentru limbaje independente de context: intersecție cu limbaje regulate, morfisme și morfisme inverse
          \end{itemize}

    \item Pe rezultat aplic lema de pompare.
          Caut un contra-exemplu de cuvânt și \(i\) pentru care lema este falsă, și astfel am o contradicție

    \item În concluzie, limbajul nu este regulat/independent de context.
\end{enumerate}

\begin{observation}
    Lemele de pompare pot fi folosite doar pentru a demonstra că un limbaj \textbf{nu} este regulat/independent de context. Pentru a demonstra afirmativa trebuie să construim automatul/gramatica corespunzătoare.
\end{observation}

\subsection*{Pentru limbaje regulate}

Fie \(\limbaj\) un limbaj regulat. Există un \(n_0 \in \naturals^*\) care depinde de limbaj, cu proprietatea că orice cuvânt \(w \in \limbaj\), cu \(\abs{w} \geq n_0\), se poate descompune în \(w = x y z\) cu proprietățile:
\begin{itemize}
    \item \(\abs{y} \geq 1\)
    \item \(\abs{xy} \leq n_0\)
    \item \(\forall i \in \naturals\), \(x y^i z \in \limbaj\)
\end{itemize}

\subsection*{Pentru limbaje independente de context}

Fie \(\limbaj\) un limbaj independent de context. Există un \(n_0 \in \naturals^*\) care depinde de limbaj, cu proprietatea că orice cuvânt \(w \in \limbaj\), cu \(\abs{w} \geq n_0\), se poate descompune în \(w = x u y v z\) cu proprietățile:
\begin{itemize}
    \item \(\abs{u v} \geq 1\)
    \item \(\abs{u y v} \leq n_0\)
    \item \(\forall i \in \naturals\), \(x u^i y v^i z \in \limbaj\)
\end{itemize}

\subsection*{Exerciții}

\begin{exercise}[exercițiul 8 din examen iunie 2011]
    Demonstrați folosind lema de pompare că limbajul
    \[\limbaj = \set{ a^{2k} b^l a^{k} | k \geq 0, l \geq 1 }\]
    nu este regulat.
\end{exercise}
\begin{proof}
    Partea din limbaj care „ne deranjează” este cea cu \(k\), unde avem o corelare a numărului de \(a\)-uri de la început cu numărul de \(a\)-uri de la sfârșit.

    Ca să simplificăm problema, vom intersecta limbajul cu un limbaj regulat.

    Presupunem că \(\limbaj\) este regulat. Atunci limbajul
    \[\limbaj' = \limbaj \cap a^* b a^* = \set{ a^{2k} b a^k | k \geq 0 }\]
    este la rândul lui regulat (am folosit o expresie regulată ca să descriu limbajul cu care intersectez).

    Fie \(n_0 \in \naturals^*\) constanta din lema de pompare pentru limbaje regulate.

    Să luăm cuvântul
    \[w = a^{2 n_0} \, b \, a^{n_0}\]
    Acesta are lungimea \(\abs{w} = 2 n_0 + 1 + n_0 = 3n_0 + 1 \geq n_0\), deci putem aplica lema pe el.

    Trebuie să ne gândim cum ar arăta o descompunere a cuvântului care să respecte proprietățile din lemă:
    \begin{align*}
        w        & = x \, y \, z \\
        \abs{y}  & \geq 1        \\
        \abs{xy} & \leq n_0
    \end{align*}

    Datorită modului în care am ales cuvântul, o descompunere care convine o să fie de forma
    \[
        w = \underbrace{aa \dots aa}_{x}
        \underbrace{aa \dots aa}_{y}
        \underbrace{aa \dots aa b aa \dots aa}_{z}
    \]
    Scris pe scurt:
    \begin{gather*}
        x = a^p \\
        y = a^q \\
        z = a^{2 n_0 - q - p} \, b \, a^{n_0}
    \end{gather*}
    unde
    \begin{align*}
        q     & \geq 1   \\
        p + q & \leq n_0
    \end{align*}

    Din lemă avem că
    \[\forall i \in \naturals, x y^i z \in \limbaj'\]
    Pentru \(i = 0\) obținem
    \begin{align*}
        x y^0 z & = a^p \, a^{2 n_0 - q - p} \, b \, a^{n_0} = \\
                & = a^{2 n_0 - q} \, b \, a^{n_0}
    \end{align*}
    Dacă aparține limbajului, ar rezulta că
    \[2 n_0 - q = 2 n_0\]
    Dar știm deja că \(q \geq 1\), deci contradicție.

    Presupunerea că \(\limbaj\) este regulat este falsă.
\end{proof}

\begin{exercise}[exercițiul 8 din examen iunie 2014]
    Demonstrați că limbajul
    \[\limbaj = \set{ a^k b^{3l} a^l | k \geq 1, l \geq 0 }\]
    nu este regulat.
\end{exercise}
\begin{proof}
    Presupunem că \(\limbaj\) este regulat.

    Putem intersecta limbajul cu \(a b^* a^*\), și obținem
    \[\limbaj' = \Set{ a b^{3l} a^l | l \geq 0 }\]

    Fie \(n_0 \in \naturals\) din lema de pompare pentru limbaje regulate.

    Putem lua cuvântul
    \[w = a \, b^{3 n_0} \, a^{n_0}\]
    care are lungimea \(\abs{w} = 1 + 3 n_0 + n_0 = 4 n_0 + 1 \geq n_0\).

    Trebuie să luăm în considerare toate descompunerile posibile pentru acest cuvânt.

    Să ne gândim cum arată cuvântul \(w\):

    \[w = \underbrace{a}_{\text{un \(a\)}} \underbrace{bbbbbbb\dots{}bbbb}_{3n_0 \text{ de \(b\)-uri}} \, \underbrace{aaa\dots{}aaa}_{n_0 \text{ de \(a\)-uri}}\]

    Știm că \(xy\) trebuie să se afle la începutul cuvântului, și din lema de pompare avem \(\abs{xy} \leq n_0\).

    \begin{enumerate}
        \item Cazul în care \(x\) îl conține pe \(a\):
              \[w = \underbrace{a bb\dots{}bbb}_{x} \underbrace{b\dots{}bb}_{y} \underbrace{bbb\dots{}bbb aa\dots{}aa}_{z}\]
              \begin{gather*}
                  x = a \, b^p \\
                  y = b^q \\
                  z = b^{3 n_0 - q - p} a^{n_0}
              \end{gather*}
              unde
              \begin{align*}
                  \abs{y}   & \geq 1 \implies q \geq 1 \\
                  \abs{x y} & \leq n_0
              \end{align*}

              Din lema de pompare avem că
              \[\forall i \in \naturals, x \, y^i \, z \in \limbaj'\]

              Luăm \(i = 0\). Obținem cuvântul
              \begin{align*}
                  x y^0 z & = a \, b^p \, b^{3 n_0 - q - p} \, a^{n_0} \\
                          & = a \, b^{3 n_0 - q} \, a^{n_0}
              \end{align*}
              care nu aparține limbajului deoarece \(3 n_0 - q < 3 n_0\).

              Contradicție cu lema de pompare, deci \(\limbaj'\) nu este regulat.

        \item Cazul în care \(x\) nu-l conține pe \(a\) (este \(\lambda\)):
              \[w = \underbrace{}_{x} \underbrace{a bb\dots{}bbb}_{y} \underbrace{bbb\dots{}bbb aa\dots{}aa}_{z}\]
              \begin{gather*}
                  x = \lambda \\
                  y = a \, b^p \\
                  z = b^{3 n_0 - p} \, a^{n_0}
              \end{gather*}
              unde
              \[1 \leq \abs{y} \leq n_0\]

              După un raționament analog ajungem la aceeași concluzie. În momentul în care ridicăm \(y\) la o putere \(i \geq 2\), o să avem mai mult de un \(a\) la început.
    \end{enumerate}

\end{proof}

\begin{exercise}[exercițiul 9 din examen iunie 2016]
    Demonstrați că limbajul
    \[\limbaj = \set{ w a^k w | w \in \set{a, b, c}^*, k \geq 0 }\]
    nu este regulat.
\end{exercise}
\begin{proof}
    Presupunem că \(\limbaj\) ar fi regulat.

    Fie \(n_0 \in \naturals\) din lema de pompare.

    Putem alege \(w = b^{n_0} \, a \, b^{n_0}\), cu \(\abs{w} = n_0 + 1 + n_0 = 2 n_0 + 1 \geq n_0\).

    Avem
    \begin{gather*}
        x = b^{p} \\
        y = b^{q} \\
        z = b^{n_0 - q - p} \, a \, b^{n_0}
    \end{gather*}
    cu
    \begin{align*}
        \abs{y}   & \geq 1   \\
        \abs{x y} & \leq n_0
    \end{align*}

    Aplicăm lema de pompare pentru \(i = 0\) și obținem
    \begin{align*}
        x y^0 z & = b^p \, b^{n_0 - q - p} \, a \, b^{n_0} \\
                & = b^{n_0 - q} \, a \, b^{n_0}
    \end{align*}
    Deoarece \(n_0 - q \neq n_0\), cuvântul nu aparține limbajului.

    Prin urmare, presupunerea noastră este falsă, limbajul nu este regulat.
\end{proof}

\begin{exercise}[exercițiul 10 din examen iunie 2017]
    Demonstrați că limbajul
    \[\limbaj = \set{ a^n b^m | n \text{ pătrat perfect}, m \geq 5 }\]
    nu este independent de context.
\end{exercise}
\begin{proof}
    Presupunem că \(\limbaj\) este independent de context.

    Îl intersectăm cu \(a^* b^5\) și obținem
    \[\limbaj' = \set{ a^n b^5 | n \text{ pătrat perfect } }\]

    Fie \(n_0 \in \naturals\) din lema de pompare.

    Alegem cuvântul
    \[w = a^{(n_0)^2} b^5\]
    care este din limbaj și are \(\abs{w} = (n_0)^2 + 5 \geq n_0\)

    Dacă luăm o descompunere în care \(u y v\) se află în zona de \(b\)-uri, atunci când o să aplicăm lema de pompare pentru \(i \neq 1\) nu mai avem \(b^5\).

    Dacă luăm o descompunere în care \(u y v\) conține și \(a\)-uri, de exemplu:
    \begin{gather*}
        x = a^p \\
        u = a^q \\
        y = a^r \\
        v = a^s \\
        z = a^{(n_0)^2 - s - r - q - p} \, b^5
    \end{gather*}
    unde
    \begin{align*}
        \abs{uv} \geq 1      & \iff q + s \geq 1       \\
        \abs{u y v} \leq n_0 & \iff q + r + s \leq n_0
    \end{align*}

    Aplicăm lema de pompare pentru un \(i\) oarecare și vedem ce obținem:
    \begin{align*}
        x u^i y v^i z & = a^p \, a^{iq} \, a^r \, a^{is} \, a^{(n_0)^2 - s - r - q - p} \, b^5 \\
                      & = a^{(n_0)^2 + q (i - 1) + s (i - 1)} \, b^5                           \\
                      & = a^{(n_0)^2 + (q + s) (i - 1)} \, b^5
    \end{align*}

    Între două pătrate perfecte nu poate exista un alt pătrat perfect. După \((n_0)^2\), următorul pătrat perfect este \((n_0 + 1)^2\), adică \((n_0)^2 + 2 n_0 + 1\).

    Dacă luăm \(i = 2\), obținem
    \[a^{(n_0)^2 + (q + s)} \, b^5\]

    Cu siguranță \(q + s \leq 2 n_0 + 1\), deci nu avem un pătrat perfect. Cuvântul nu aparține limbajului.

    În concluzie, \(\limbaj\) nu este independent de context.
\end{proof}

\begin{exercise}[exercițiul bonus din examen iunie 2013]
    Demonstrați că limbajul
    \[\limbaj = \set{ 0^n 1^m | n > 0, m \text{ prim} }\]
    nu este independent de context.
\end{exercise}
\begin{proof}
    Presupunem că \(\limbaj\) este independent de context.

    Îl putem intersecta cu \(0 1^*\) și obținem
    \[\limbaj = \set{ 0 1^m | m \text{ prim } }\]

    Fie \(n_0 \in \naturals\) din lema de pompare.

    Notăm cu \(P\) = următorul număr prim mai mare decât \(n_0\).
    Cuvântul \(w = 0 1^P\) aparține limbajului și are lungime \(\abs{w} = 1 + P \geq n_0\).

    Avem mai multe cazuri, în funcție de cum este descompunerea:
    \begin{enumerate}
        \item
              \begin{gather*}
                  x = 0 \, 1^p \\
                  u = 1^q \\
                  y = 1^r \\
                  v = 1^s \\
                  z = 1^{P - s - r - q - p}
              \end{gather*}
              unde
              \begin{align*}
                  \abs{u v} \geq 1     & \iff q + s \geq 1       \\
                  \abs{u y v} \leq n_0 & \iff q + r + s \leq n_0
              \end{align*}

              Aplicăm lema de pompare pentru un \(i \in \naturals\) oarecare:
              \begin{align*}
                  x u^i y v^i z & = 0 \, 1^p \, 1^{iq} \, 1^r \, 1^{is} \, 1^{P - s - r - q - p} \\
                                & = 0 \, 1^{P + q(i - 1) + s(i - 1)}                             \\
                                & = 0 \, 1^{P + (q + s)(i - 1)}
              \end{align*}

              Pentru \(i = P + 1\), obținem cuvântul
              \[0 \, 1^{P + (q + s) P} = 0 \, 1^{(q + s)(P + 1)}\]
              care nu aparține limbajului, deoarece exponentul lui \(1\) este număr compus. Contradicție cu lema de pompare.

        \item
              \begin{gather*}
                  x = \lambda \\
                  u = 0 \, 1^p \\
                  y = 1^q \\
                  v = 1^r \\
                  z = 1^{P - r - q - p}
              \end{gather*}

              Aplicând lema de pompare obținem un rezultat analog cu primul caz.

        \item
              \begin{gather*}
                  x = u = \lambda \\
                  y = 0 \, 1^p \\
                  v = 1^q \\
                  z = 1^{P - q - p}
              \end{gather*}

              Asemănător cu cazurile de mai sus.

        \item
              \begin{gather*}
                  x = u = y = \lambda \\
                  v = 0 \, 1^p \\
                  z = 1^{P - p}
              \end{gather*}

              Asemănător cu cazurile de mai sus.
    \end{enumerate}

    În concluzie, \(\limbaj\) nu este independent de context.
\end{proof}

\begin{exercise}[exercițiul 11 din examen iunie 2011]
    Demonstrați că limbajul
    \[\limbaj = \set{ a^i b^j c^k | i < j \text{ și } i + 2j + 3 < k }\]
    nu este independent de context.
\end{exercise}
\begin{proof}
    Presupunem că \(lang\) este independent de context.

    Fie \(n_0\) din lema de pompare.

    Vrem să alegem un cuvânt în care inegălitățile să fie satisfăcute cât mai la limită.
    Alegem \(i = n_0\), din \(i < j\) putem alege \(j = n_0 + 1\).
    Din \(i + 2j + 3 < k\), alegem \(k = 3 n_0 + 6\).

    Cuvântul ales este \(w = a^{n_0} \, b^{n_0 + 1} \, c^{3 n_0 + 6}\), unde \(\abs{w} = 5 n_0 + 7 \geq n_0\).

    Obținem foarte multe descompuneri posibile:

    Dacă
    \begin{gather*}
        x = a^p \\
        u = a^q \\
        y = a^r \\
        v = a^s \\
        z = a^{n_0 - s - r - q - p} \, b^{n_0 + 1} \, c^{3 n_0 + 6}
    \end{gather*}
    putem pompa cu \(i > 1\) și vom obține un număr de \(a\)-uri care depășește numărul de \(b\)-uri, ieșind din limbaj.

    Dacă
    \begin{gather*}
        x = a^{n_0} b^p \\
        u = b^q \\
        y = b^r \\
        v = b^s \\
        z = b^{n_0 + 1 - s - r - q - p} c^{3 n_0 + 6}
    \end{gather*}
    putem pompa cu \(i > 1\) și vom depăși a doua condiție.

    etc.
\end{proof}

\begin{exercise}[exercițiul 10 din examen iunie 2016]
    Demonstrați că limbajul
    \[\limbaj = \set{ a^n b^m c^r | n \geq m \geq r \geq 150 }\]
    nu este independent de context.
\end{exercise}
\begin{proof}
    Limbajul se poate rescrie ca
    \[\limbaj = \set{ a^n b^m c^r | n \geq m, m \geq r, r \geq 150 }\]

    Alegem un cuvânt cum ar fi \(w = a^{n_0} b^{n_0} c^{n_0}\) și aplicăm lema de pompare.

    \begin{itemize}
        \item În cazurile în care pompăm doar \(a\)-uri putem lua \(i = 0\) și o să avem mai puține \(a\)-uri decât \(b\)-uri.
        \item Când pompăm și \(a\)-uri și \(b\)-uri sau doar \(b\)-uri putem lua \(i = 0\) și avem mai puține \(b\)-uri decât \(c\)-uri.
        \item Când pompăm \(c\)-uri putem lua un \(i > 1\) și o să avem mai multe \(c\)-uri decât \(b\)-uri.
    \end{itemize}
\end{proof}

\section*{Teorie la examen}

Prima jumătate a examenului se bazează pe teoria învățată la curs.

La primele două probleme se cere demonstrația a unor teoreme/propoziții din curs.

Urmează câte 4 afirmații a căror valoare de adevăr trebuie determinată, cu justificări (pe scurt și la obiect). Acestea nu apar explicit în curs, dar se bazează pe observațiile și informațiile din materiale.

\subsection*{Decidabilitate}

Ca să arăți că o anumită problemă este \emph{decidabilă}, trebuie să arăți că există un algoritm pentru asta. De exemplu, să verifici că un număr e prim este decidabil, pentru că îl poți descompune în factori și vezi să nu aibă alți factori în afară de el însuși.

Ca să arăți că o anumită problemă \emph{nu este decidabilă}, trebuie să te bazezi pe afirmații din curs. De exemplu, se știe că nu există un algoritm pentru a găsi intersecția a două gramatici independente de context.

\subsection*{Egalitatea de limbaje}

Ca să verifici dacă două limbaje regulate \(\limbaj_1, \limbaj_2\) sunt egale, construiești automate finite care să accepte \(\limbaj_1\) respectiv \(\limbaj_2\), apoi aplici algoritmul din curs pentru a verifica că automatele sunt echivalente.

Pentru limbaje independente de context, nu se poate verifica (nu este decidabilă) egalitatea.

\subsection*{Exerciții de teorie}

Pentru fiecare dintre următoarele exerciții, cerința este să se \textbf{justifice} dacă afirmația respectivă este \emph{adevărată} sau \emph{falsă}.

\subsubsection*{Examen 2011}

\begin{exercise}
    Există limbaje regulate care nu sunt independente de context.
\end{exercise}
\begin{proof}
    Afirmația este falsă, limbajele regulate sunt incluse strict în cele independente de context, conform \href{https://en.wikipedia.org/wiki/Chomsky_hierarchy}{ierarhiei Chomsky}.

    Pentru orice limbaj regulat putem scrie o gramatică regulată corespunzătoare, care este inclusă în gramaticile independente de context, deci și limbajul este inclus în limbajele independente de context.
\end{proof}

\begin{exercise}
    Fie limbajele \(\limbaj_1, \limbaj_2\) cu proprietatea că  \(\limbaj_1 \subseteq \limbaj_2\) și \(\limbaj_2 \in REG\). Atunci \(\limbaj_1 \in REG\).
\end{exercise}
\begin{proof}
    Fals, un contraexemplu este să luăm \(\limbaj_1 = \set{ a^n b^n | n \in \naturals }\), \(\limbaj_2 = \set{ a, b }^*\).

    Avem că \(\limbaj_1 \subseteq \limbaj_2\) dar știm că \(\limbaj_1\) nu este regulat.
\end{proof}

\begin{exercise}
    Este decidabil dacă limbajele acceptate de doua automate finite deterministe sunt egale sau nu.
\end{exercise}
\begin{proof}
    Dacă avem două automate finite deterministe, putem aplica algoritmul de minimizare din curs (cel bazat pe un tabel). În urma acestui algoritm putem determina dacă automatele sunt echivalente.
\end{proof}

\begin{exercise}
    Este decidabil dacă limbajul intersecției a două gramatici independente de context este vid sau nu.
\end{exercise}
\begin{proof}
    Nu există un algoritm pentru determinarea intersecției a două gramatici independente de context. De aceea, nu se poate verifica dacă aceasta este vidă sau nu.
\end{proof}

\subsubsection*{Examen 2013}

\begin{exercise}
    Există o gramatică regulată \(G\) astfel încât nu există niciun automat finit nedeterminist (\nfa) care să accepte exact \(\limbaj(G)\).
\end{exercise}
\begin{proof}
    Afirmația este falsă. Prin definiție, gramaticile \emph{regulate} definesc un limbaj regulat. Pentru orice limbaj regulat se poate scrie un \nfa{}.
\end{proof}

\begin{exercise}
    Este decidabil dacă limbajele acceptate de două automate finite nedeterministe cu lambda mișcări sunt egale sau nu.
\end{exercise}
\begin{proof}
    Orice \lnfa{} poate fi redus la un \dfa{}, deci această întrebare este echivalentă cu exercițiul 3.
\end{proof}

\begin{exercise}
    Fie limbajele \(\limbaj_1, \limbaj_2\) cu proprietatea că \(\limbaj_2 \subseteq \limbaj_1\) și \(\limbaj_2 \in CF\). Atunci \(\limbaj_1 \in CF\).
\end{exercise}
\begin{proof}
    Putem avea \(\limbaj_2 = \set{ a^n b^n | n \in \naturals }\) care știm că este independent de context, și \(\limbaj_1 = \limbaj_2 \cup \set{ c^p | p \text{ prim} }\).

    Avem că \(\limbaj_2 \subseteq \limbaj_1\) dar \(\limbaj_1 \not\in CF\).
\end{proof}

\begin{exercise}
    Există limbaje finite care nu sunt regulate.
\end{exercise}
\begin{proof}
    Dacă avem un limbaj finit format din cuvintele \(\set{ w_1, \dots, w_n }\), putem să scriem o expresie regulată \(w_1 + \dots + w_n\) care să îl accepte.
\end{proof}

\subsubsection*{Examen 2014}

\begin{exercise}
    Există o gramatică regulată \(G\) astfel încât nu există nicio expresie regulată \(E\) cu proprietatea că \(\limbaj(E) = \limbaj(G)\).
\end{exercise}
\begin{proof}
    La fel ca la exercițiul 5, gramaticile regulate și expresiile regulate definesc aceeași familie de limbaje, limbajele regulate.
\end{proof}

\begin{exercise}
    Fie limbajele \(\limbaj_1, \limbaj_2\) cu proprietatea că \(\limbaj_1 \subseteq \limbaj_2\) și \(\limbaj_1 \in REG\). Deci \(\limbaj_2 \in REG\).
\end{exercise}
\begin{proof}
    Fals, putem da un contraexemplu ca la exercițiul 7, cu \(\limbaj_1 = \set{ a^n | n  }\) și \(\limbaj_2 = \limbaj_1 \cup \set{ b^p | p \text{ prim}}\).
\end{proof}

\begin{exercise}
    Este decidabil dacă limbajele acceptate de două expresii regulate sunt egale.
\end{exercise}
\begin{proof}
    Expresiile regulate pot fi transformate în automate finite. Putem verifica echivalența automatelor finite ca la exercițiul 3.
\end{proof}

\begin{exercise}
    Există limbaje modelate de automate push-down deterministe care au toate cuvintele de lungime impară și nu pot fi modelate de gramatici independente de context.
\end{exercise}
\begin{proof}
    Nu există, limbajele acceptate de \emph{orice fel} de automat push-down sunt incluse în limbajele modelate de gramaticile independente de context.
\end{proof}

\subsubsection*{Examen 2016}

\begin{exercise}
    Fie limbajele \(\limbaj_1, \limbaj_2, \limbaj_3\) cu proprietatea că \(\limbaj_1 \cup \limbaj_2 = \limbaj_3\) și \(\limbaj_2, \limbaj_3 \in REG\). Deci \(\limbaj_1 \in REG\).
\end{exercise}
\begin{proof}
    Fals, putem avea \(\limbaj_2 = \limbaj_3 = \set{ a, b }^*\), și \(\limbaj_1 = \set{ a^n b^n | n \in \naturals}\), care nu este regulat.
\end{proof}

\begin{exercise}
    Există o gramatică regulată \(G\) peste alfabetul \(\set{ a, b, c }\) astfel încât nu există niciun \nfa{} \(A\) cu proprietatea că
    \[\limbaj(A) = \limbaj(G) \cup \set{ acccab, bbaabb }\]
\end{exercise}
\begin{proof}
    Limbajul modelat de \(G\) este unul regulat, deoarece gramatica este regulată. De asemenea \(\set{ acccab, bbaabb }\) este o mulțime finită de cuvinte, deci este un limbaj finit, deci limbaj regulat.

    Reuniunea a două limbaje regulate este tot un limbaj regulat. Deci termenul din dreapta este sigur acceptabil de un \nfa{}.
\end{proof}

\begin{exercise}
    Există limbaje modelate de gramatici independente de context care au toate cuvintele de lungime impară și nu pot fi modelate de automate push-down deterministe?
\end{exercise}
\begin{proof}
    Un exemplu de limbaj care nu poate fi acceptat de automate push-down \emph{deterministe} este \(\set{ w w^R | w \in \set{a, b}^* }\), unde \(R\) înseamnă oglinditul cuvântului.

    Toate cuvintele din acest limbaj au lungime \(\abs{w} + \abs{w^R} = 2 \abs{w}\), care este un număr par. Trebuie să mai adăugăm o literă ca lungimea să fie impară.

    Dacă adăugăm o literă distinctivă în mijloc (de exemplu un \(c\)), limbajul ar putea fi acceptat și de automate push-down deterministe. Soluția este să adăugăm litera la început sau la final:
    \[\limbaj = \set{ w w^R c | w \in \set{a, b}^* }\]
\end{proof}

\begin{exercise}
    Este decidabil dacă limbajele acceptate de un \nfa{} cu lambda mișcări și o gramatică regulată cu cel mult 20 de producții sunt egale.
\end{exercise}
\begin{proof}
    Orice gramatică regulată produce un limbaj regulat, și putem verifica echivalența de limbaje regulate ca la exercițiul 3.
\end{proof}

\subsubsection*{Examen 2017}

\begin{exercise}
    Fie limbajele \(\limbaj_1, \limbaj_2, \limbaj_3\) cu proprietatea că \(\limbaj_1 \cup \limbaj_2 = \limbaj_3\) și \(\limbaj_2, \limbaj_3 \in REG\). Deci \(\limbaj_1 \in REG\).
\end{exercise}
\begin{proof}
    Identic cu exercițiul 13.
\end{proof}

\begin{exercise}
    Există o gramatică neambiguă \(G\) peste alfabetul \(\set{ a, b, c }\) astfel încât nu există niciun \nfa{} \(A\) cu proprietatea că
    \[\limbaj(A) = \limbaj(G) \cup \set{ acccab, bbaabb }\]
\end{exercise}
\begin{proof}
    O gramatică \emph{neambiguă} este independentă de context, nu neapărat regulată. Putem să luăm contraexemplu gramatica
    \begin{align*}
        S & \rightarrow aSb     \\
        S & \rightarrow \lambda
    \end{align*}
    care este neambiguă și generează \(\set{ a^n b^n | n \in \naturals }\). Aceasta nu poate fi acceptată de un automat finit.
\end{proof}

\begin{exercise}
    Există limbaje modelate de gramatici independente de context care au toate cuvintele de lungime impară și nu pot fi modelate de automate push-down cu acceptare prin stare finală și stivă vidă.
\end{exercise}
\begin{proof}
    Fals, orice gramatică independentă de context poate fi modelată de un automat push-down. Dacă folosim automate push-down nedeterministe atunci modul de acceptare nu contează.
\end{proof}

\begin{exercise}
    Este decidabil dacă limbajele decidabile de o expresie regulată cu cel puțin 20 de operator și o gramatică regulată cu cel puțin 20 de producții sunt egale.
\end{exercise}
\begin{proof}
    Limbajele produse de expresiile regulate și de gramaticile regulate sunt limbaje regulate, deci putem să le verificăm echivalența ca la exercițiul 3.
\end{proof}

\subsubsection*{Examen 2018}

\begin{exercise}
    Este decidabil dacă limbajele acceptate de un \nfa{} cu lambda mișcări și un automat push-down aciclic sunt egale.
\end{exercise}
\begin{proof}
    Dacă automatul push-down este aciclic, atunci el nu poate produce o infinitate de cuvinte; dacă facem o mulțime cu toate cuvintele pe care le produce, aceasta va fi finită.

    Combinând asta cu concluzia de la exercițiul 8, care ne spune că orice limbaj (mulțime de cuvinte) finit este regulat, avem că limbajul acceptat de acest automat push-down este \emph{regulat}. Deci putem construi un automat finit care să îl accepte, și apoi să verificăm echivalența între acesta și un \nfa{}.
\end{proof}

\begin{exercise}
    Fie limbajele \(\limbaj_1, \limbaj_2, \limbaj_3\) cu proprietatea că \(\limbaj_1 \cap \limbaj_2 = \limbaj_3\) și \(\limbaj_1, \limbaj_2 \in CFL\). Deci \(\limbaj_3 \in CFL\).
\end{exercise}
\begin{proof}
    Fals, limbajele independente de context nu sunt închise la intersecție. Contraexemplul clasic este \(\limbaj_1 = \set{ a^n b^n c^i | n \in \naturals, i \in \naturals }\), \(\limbaj_2 = \set{ a^j b^m c^m | j \in \naturals, m \in \naturals }\).

    \(\limbaj_3\) nu este independent de context:
    \[\limbaj_1 \cap \limbaj_2 = \set{ a^n b^n c^n | n \in \naturals } = \limbaj_3\]
\end{proof}

\begin{exercise}
    Fie limbajele \(\limbaj_1, \limbaj_2, \limbaj_3\) cu proprietatea că \(\limbaj_1 \cap \overline{\limbaj_2} = \limbaj_3\) și \(\limbaj_2, \limbaj_3 \in REG\). Deci \(\limbaj_1 \in REG\).
\end{exercise}
\begin{proof}
    Fals, un contraexemplu este \(\limbaj_1 = \set{ a^n b^n | n \in \naturals }\), \(\limbaj_2 = \set{a, b}^*\), \(\limbaj_3 = \varnothing\).

    Avem că
    \[\limbaj_1 \cap \overline{\limbaj_2} = \limbaj_1 \cap \overline{\set{ a, b }^* } = \limbaj_1 \cap \varnothing = \varnothing = \limbaj_3\]
\end{proof}

\begin{exercise}
    Fie limbajele \(\limbaj_1, \limbaj_2, \limbaj_3\) cu proprietatea că \(\limbaj_1 \setminus \limbaj_2 = \limbaj_3\) și \(\limbaj_2, \limbaj_3 \in REG\). Deci \(\limbaj_1 \in REG\).
\end{exercise}
\begin{proof}
    Deoarece \(A \setminus B = A \cap \overline{B}\), această problemă este echivalentă cu exercițiul 23.
\end{proof}

\section*{Automate Pushdown}

Un \emph{automat pushdown} este un automat finit extins cu o stivă de dimensiune infinită. Acesta operează la fel ca un automat obișnuit, prelucrând cuvintele literă cu literă, dar poate să stocheze și să ia decizii pe baza memoriei sale.

Limbajele acceptate de \pda{} sunt aceleași cu limbajele generate de \textbf{gramatici independente de context}.

Automatele pushdown deterministe sunt incluse \textbf{strict} în cele nedeterministe.
De exemplu, se poate demonstra că limbajul
\[\lang = \set{ w w^R | w \in \set{a, b}^* }\]
nu poate fi acceptat de un \dpda{}, dar este destul de simplu să facem un \npda{} pentru el.

Vom folosi simbolul \(Z_0\) pentru a indica o stivă vidă: în momentul în care \pda{}-ul începe execuția, presupunem că acest simbol se află pe stivă, și îl vom lăsa tot timpul la baza stivei.

\subsection*{Exerciții}

\begin{exercise}[exercițiul 10 din examen iunie 2011]
    Construiți un automat pushdown (\pda{}), eventual determinist, pentru limbajul
    \[
        \lang = \set{ w | w \in \set{a, b, c}^*, \abs{w}_a = \abs{w}_b + 2 }
    \]
\end{exercise}

\textit{Idee}: Pe stivă reținem dacă am citit mai multe \(a\)-uri sau mai multe \(b\)-uri până la acel moment. La finalul cuvântului, trebuie să avem două \(a\)-uri în plus.

\begin{proof}~
    \begin{figure}[H]
        \centering
        \caption*{Construim următorul \pda{}, cu acceptare prin stare finală și stivă vidă:}
        \begin{tikzpicture}
            \node[state, initial] (1) {\(q_1\)};
            \node[state, right of=1] (2) {\(q_2\)};
            \node[state, right of=2] (3) {\(q_3\)};
            \node[state, accepting, right of=3] (4) {\(q_4\)};

            \draw (1) edge[above] node[align=center] {
                    \(\lambda, Z_0/Z_0\) \\
                    \(\lambda, A/A\) \\
                    \(\lambda, B/B\)
                } (2)
            (1) edge[loop above] node[align=center] {
                    \(a, Z_0/AZ_0\) \\
                    \(a, A/AA\) \\
                    \(a, B/\lambda\) \\
                    \(b, Z_0/BZ_0\) \\
                    \(b, B/BB\) \\
                    \(b, A/\lambda\)
                } (1)
            (2) edge[above] node {\(\lambda, A/\lambda\)} (3)
            (3) edge[above] node {\(\lambda, A/\lambda\)} (4);
        \end{tikzpicture}
    \end{figure}

    În starea \(q_1\), automatul ciclează, citește caractere și le pune pe stivă. La un moment dat, acesta trece nedeterminist în \(q_2\). Tranzițiile de la \(q_2\) la \(q_3\) și de la \(q_3\) la \(q_4\) verifică că avem două \(a\)-uri în plus pe stivă.
\end{proof}

\begin{exercise}[exercițiul 11 din examen iunie 2013]
    Construiți un automat pushdown (\pda{}), eventual determinist, pentru limbajul
    \[
        \lang = \set{ w | w \in \set{0, 1, 2}^*, 4\abs{w}_0 + 1 = \abs{w}_2 }
    \]
\end{exercise}

\textit{Idee}: La fel ca mai sus, dar punem de 4 ori mai multe simboluri pentru \(0\) decât pentru \(2\). La final trebuie să rămână un \(2\) în plus.

\begin{exercise}[exercițiul 10 din examen iunie 2014]
    Construiți un automat pushdown (\pda{}) pentru limbajul
    \[
        \lang = \set{ a^n b^n | n \geq 0 } \cup \set{ a^n b^{3n} | n \geq 0 }
    \]
\end{exercise}

\textit{Idee}: Reținem câte \(a\)-uri citim pe stivă. Când dăm de primul \(b\), automatul se bifurcă nedeterminist: pe un caz o să corelăm un \(b\) cu un \(a\), pe altul câte 3 \(b\)-uri cu un \(a\).

\begin{proof}~
    \begin{figure}[H]
        \centering
        \caption*{Construim următorul \pda{}, cu acceptare prin stare finală și stivă vidă:}
        \begin{tikzpicture}
            \node[state, initial] (1) {\(q_1\)};

            \node[state, above right of=1] (2) {\(q_2\)};
            \node[state, accepting, right of=2] (3) {\(q_3\)};

            \node[state, below right of=1] (4) {\(q_4\)};
            \node[state, right of=4] (5) {\(q_5\)};
            \node[state, right of=5] (6) {\(q_6\)};;

            \draw (1) edge[loop above] node[align=center] {
                    \(a, Z_0/AZ_0\) \\
                    \(a, A/AA\)
                } (1)
            (1) edge[below right] node {\(b, A/\lambda\)} (2)
            (1) edge[above right] node {\(b, A/\lambda\)} (4)

            (2) edge[loop above] node[align=center] {
                    \(b, A/\lambda\)
                } (3)
            (2) edge[above] node {\(\lambda, Z_0/Z_0\)} (3)

            (4) edge[above] node[align=center] {
                    \(b, Z_0/Z_0\) \\
                    \(b, A/A\)
                } (5)
            (5) edge[above] node[align=center] {
                    \(b, Z_0/Z_0\) \\
                    \(b, A/A\)
                } (6)
            (6) edge[below, bend left] node {\(b, A/\lambda\)} (4)
            (6) edge[right, bend right] node {\(\lambda, Z_0/Z_0\)} (3);
        \end{tikzpicture}
    \end{figure}

    În starea \(q_1\), automatul citește \(a\)-uri și le pune pe stivă. La primul \(b\) citit, trece nedeterminist în starea \(q_2\) sau \(q_4\).

    În starea \(q_2\) citește \(b\)-uri și le corelează cu \(a\)-uri.

    În stările \(q_4\), \(q_5\) și \(q_6\) avem un ciclu de lungime 3 care corelează câte 3 \(b\)-uri cu un \(a\).
\end{proof}

\begin{exercise}[exercițiul 11 din examen iunie 2017]
    Construiți un automat pushdown (\pda{}) cu acceptare prin stivă vidă pentru limbajul
    \[
        \lang = \set{ w | w \in \set{a, b}^*, 2 \abs{w}_a \neq 3 \abs{w}_b + 2 } \cup \set{ aaab, bbba }
    \]
\end{exercise}

\textit{Idee}: Rezolvăm ca mai sus, reținând câte \(a\)-uri / \(b\)-uri am citit pe stivă. La final verificăm să nu se îndeplinească egalitatea.

\begin{exercise}[exercițiul 11 din examen iunie 2018]
    Construiți un automat pushdown (\pda{}) pentru limbajul
    \[
        \lang = \set{ w w^R c^i | w \in \set{a, b}^*, i \geq 2 } \cup \set{ abca, ccba, cabc }
    \]
    unde \(w^R\) înseamnă oglinditul lui \(w\), de exemplu \((abcd)^R = dcba\).
\end{exercise}

\textit{Idee}: Reținem pe stivă literele lui \(w\), și alegem nedeterminist care este momentul în care începe \(w^R\).

\begin{proof}~
    \begin{figure}[H]
        \centering
        \caption*{Construim următorul \pda{} nedeterminist:}
        \begin{tikzpicture}
            \node[state, initial] (0) {\(q_0\)};
            \node[state, below right of=0] (dfa) {\(q'\)};
            \coordinate[right of=dfa] (dfa_body);

            \node[state, above right of=0] (1) {\(q_1\)};
            \node[state, right of=1] (2) {\(q_2\)};
            \node[state, right of=2] (3) {\(q_3\)};
            \node[state, accepting, right of=3] (4) {\(q_4\)};

            \draw (0) edge[below left] node {\(\lambda, Z_0/Z_0\)} (dfa)
            (dfa) edge[dotted] (dfa_body)
            (0) edge[above left] node {\(\lambda, Z_0/Z_0\)} (1)
            (1) edge[loop above] node[align=center] {
                    \(a, Z_0/AZ_0\) \\
                    \(a, A/AA\) \\
                    \(a, B/AB\) \\
                    \(b, Z_0/BZ_0\) \\
                    \(b, A/BA\) \\
                    \(b, B/BB\)
                } (1)
            (1) edge[above] node[align=center] {
                    \(\lambda, Z_0/Z_0\) \\
                    \(\lambda, A/A\) \\
                    \(\lambda, B/B\)
                } (2)
            (2) edge[loop above] node[align=center] {
                    \(a, A/\lambda\) \\
                    \(b, B/\lambda\)
                } (2)
            (2) edge[above] node {\(c, Z_0/Z_0\)} (3)
            (3) edge[above] node {\(c, Z_0/Z_0\)} (4)
            (4) edge[loop above] node {\(c, Z_0/Z_0\)} (4);
        \end{tikzpicture}
    \end{figure}

    În \(q_1\) punem pe stivă câte un simbol pentru fiecare literă citită. Alegem nedeterminist care este mijlocul cuvântului, trecând în starea \(q_2\). Aici citim literele și comparăm cu ce avem pe stivă.

    În starea \(q_4\) acceptăm oricâte \(c\)-uri; am verificat deja că avem cel puțin două prin tranzițiile de la \(q_2\) la \(q_3\) și de la \(q_3\) la \(q_4\).

    Începând cu starea \(q'\) avem \dfa{}-ul pentru mulțimea \(\set{ abca, ccba, cabc }\), pe care nu l-am mai scris.
\end{proof}

\section*{Minimizarea și echivalența automatelor finite}

Ca să minimizăm un \dfa{}, eliminăm stările \emph{inaccesibile} și le grupăm pe cele \emph{redundante}.

O stare este \textbf{inaccesibilă} dacă nu există niciun drum de la starea inițială la ea. Aceste stări pot fi eliminate fără problemă dintr-un automat, nu modifică limbajul acceptat de acesta.
\begin{figure}[H]
    \centering
    \caption*{Exemplu de stări inaccesibile (\(q_A\) și \(q_B\))}
    \begin{tikzpicture}
        \node[state, initial] (0) {\(q_0\)};
        \node[state, right of=0] (1) {\(q_1\)};
        \node[state, right of=1, accepting] (2) {\(q_2\)};

        \node[state, above left of=1] (a) {\(q_A\)};
        \node[state, above right of=1] (b) {\(q_B\)};

        \draw (0) edge (1)
        (1) edge (2)
        (a) edge (1);
    \end{tikzpicture}
\end{figure}

Două stări sunt \textbf{echivalente} (nu pot fi distinse) dacă din ele se acceptă exact aceleași limbaje. Aceste stări pot fi unite într-una singură.
\begin{figure}[H]
    \centering
    \caption*{Exemplu de stări echivalente (\(q_A\) și \(q_B\))}
    \begin{tikzpicture}
        \coordinate (pre_top);
        \node[state, right of=pre_top] (top_left) {\(q_A\)};
        \node[state, right of=top_left] (top_right) {\(q_0\)};

        \coordinate[below of=pre_top] (pre_bottom);
        \node[state, below of=top_right] (bottom) {\(q_B\)};

        \node[state, below right of=top_right] (center) {\(q_1\)};
        \coordinate[right of=center] (after_center);

        \draw (pre_top) edge[dotted] (top_left)
        (top_left) edge[above] node {\(a\)} (top_right)
        (top_right) edge[below left] node {\(\lambda\)} (center)

        (pre_bottom) edge[dotted] (bottom)
        (bottom) edge[below right] node {\(a\)} (center)

        (center) edge[dotted] (after_center);
    \end{tikzpicture}
\end{figure}

Asta ne permite să determinăm iterativ care sunt stările echivalente.

\subsection*{Exemplu}

Să se minimizeze următorul DFA:
\begin{figure}[H]
    \centering
    \begin{tikzpicture}
        \node[state, initial] (A) {\(A\)};
        \node[state, above right of=A] (B) {\(B\)};
        \node[state, below right of=A] (C) {\(C\)};
        \node[state, right of=B] (D) {\(D\)};
        \node[state, accepting, right of=C] (E) {\(E\)};


        \draw (A) edge[above left] node {\(0\)} (B)
        (A) edge[below left] node {\(1\)} (C)

        (B) edge[loop above] node {\(0\)} (B)
        (B) edge[bend right, below] node {\(1\)} (D)

        (C) edge[left] node {\(0\)} (B)
        (C) edge[loop below] node {\(1\)} (C)

        (D) edge[bend right, above] node {\(0\)} (B)
        (D) edge[right] node {\(1\)} (E)

        (E) edge[below left] node {\(0\)} (B)
        (E) edge[below] node {\(1\)} (C);
    \end{tikzpicture}
\end{figure}
\begin{proof}
    O rezolvare pas cu pas se poate găsi în \href{https://www.youtube.com/watch?v=Dx2RJ2DXRYs}{acest video}.
\end{proof}

\section*{Forma Normală Chomsky (FNC)}

În comparație cu automatele finite, la gramatici \textbf{nu} există o unică reprezentare minimă. Putem obține mai multe forme Chomsky pornind de la aceeași gramatică, și nu putem să le comparăm în vreun fel.

O gramatică se află în \textbf{formă normală Chomsky} dacă toate producțiile ei sunt de forma
\begin{itemize}
    \item \(A \to a\), unde \(a\) este un terminal
    \item \(A \to XY\), unde \(X\) și \(Y\) sunt neterminale
\end{itemize}

\subsection*{Algoritm de obținere a FNC}
Aceasta este ordinea \href{https://en.wikipedia.org/wiki/Chomsky_normal_form#Order_of_transformations}{recomandată} pentru că limitează numărul de producții care apar în final (altfel, putem avea un număr \emph{exponențial} de producții).
\begin{enumerate}
    \item Eliminăm simbolurile cu o buclă care nu se termină. De exemplu, producțiile
          \begin{align*}
              A & \to B \\
              B & \to C \\
              C & \to A
          \end{align*}
          pot fi eliminate.

    \item Eliminăm simbolurile care nu pot fi obținute plecând din simbolul de start. De exemplu, din gramatica
          \begin{align*}
              S & \to A \\
              A & \to a \\
              X & \to b
          \end{align*}
          putem elimina \(X \to b\).

    \item Înlocuim terminalii din producți cu neterminali. Dacă avem, de exemplu:
          \[A \to AbCDe\]
          transformăm în
          \begin{align*}
              A & \to ABCDE \\
              B & \to b     \\
              E & \to e
          \end{align*}

    \item Eliminăm producțiile de lungime \(> 2\). Orice producție lungă poate fi spartă în mai multe producții mai mici. De exemplu:
          \[A \to BCDE\]
          devine:
          \begin{align*}
              A & \to BX \\
              X & \to CY \\
              Y & \to DE
          \end{align*}
          unde toate producțiile au lungime 2.

    \item Eliminăm \(\lambda\)-producțiile. Dacă un neterminal are o producție care merge în \(\lambda\), eliminăm această producție, căutăm toate celelalte producții care îl conțin,
          \begin{align*}
              A & \to XaYb           \\
              X & \to x \mid \lambda \\
              Y & \to y \mid \lambda
          \end{align*}
          devine
          \begin{align*}
              A & \to XaYb \mid aYb \mid Xab \mid ab \\
              X & \to x                              \\
              Y & \to y
          \end{align*}

          Dacă într-o producție avem \(n\) neterminali care pot genera \(\lambda\), o vom înlocui cu \(2^n\) producții.

          \begin{observation}
              Prin acest proces este posibil să-l pierdem pe \(\lambda\) din limbajul generat de gramatică. Obținem ce se numește o gramatică \(\lambda\)-echivalentă cu cea inițială (aceleași cuvinte, dar lipsește \(\lambda\)).
          \end{observation}

    \item Eliminăm producțiile unitate. Producțiile unitate sunt de forma
          \[A \to B\]
          Eliminăm această producție, și înlocuim peste tot unde era \(B\) cu \(A\) (sau invers).
\end{enumerate}

\subsection*{De ce este utilă această formă?}

Un \emph{parser} este un program care poate determina un cuvânt \textbf{aparține sau nu unei gramatici}, și dacă aparține, \textbf{cum se poate deriva} acesta.
Un exemplu de algoritm care folosește FNC pentru a determina arborele de derivare este \href{https://en.wikipedia.org/wiki/CYK_algorithm}{algoritmul Cocke-Younger-Kasami (CYK)}.

\subsection*{Exemplu}

Să se reducă la Formă Normală Chomsky următoarea gramatică:
\begin{align*}
    S & \to T \mid U \mid X             \\
    T & \to VaT \mid VaV \mid TaV       \\
    U & \to VbU \mid VbV \mid UbV       \\
    V & \to aVbV \mid bVaV \mid \lambda \\
    X & \to Y                           \\
    Y & \to X
\end{align*}
Aceasta este inspirată de cea care generează \href{https://en.wikipedia.org/wiki/Context-free_grammar#Distinct_number_of_a's_and_b's}{cuvintele cu un număr diferit de \(a\)-uri și \(b\)-uri}.

\begin{proof}
    Aplicăm pașii de mai sus:
    \begin{enumerate}
        \item Eliminăm ultimele două producții, care ciclează:
              \begin{align*}
                  S & \to T \mid U \mid X             \\
                  T & \to VaT \mid VaV \mid TaV       \\
                  U & \to VbU \mid VbV \mid UbV       \\
                  V & \to aVbV \mid bVaV \mid \lambda \\
              \end{align*}

        \item Eliminăm simbolul nefolosit \(X\):
              \begin{align*}
                  S & \to T \mid U                    \\
                  T & \to VaT \mid VaV \mid TaV       \\
                  U & \to VbU \mid VbV \mid UbV       \\
                  V & \to aVbV \mid bVaV \mid \lambda
              \end{align*}

        \item Extragem neterminalii \(a\) și \(b\):
              \begin{align*}
                  S & \to T \mid U                    \\
                  T & \to VAT \mid VAV \mid TAV       \\
                  U & \to VBU \mid VBV \mid UBV       \\
                  V & \to AVBV \mid BVAV \mid \lambda \\
                  A & \to a                           \\
                  B & \to b
              \end{align*}

              \newpage{}

        \item Spargem producțiile lungi în mai multe producții scurte, introducând noi neterminali:
              \begin{align*}
                  S   & \to T \mid U                     \\
                  T   & \to V T_1 \mid V T_2 \mid T T_3  \\
                  T_1 & \to AT                           \\
                  T_2 & \to AV                           \\
                  T_3 & \to AV                           \\
                  U   & \to V U_1 \mid V U_2 \mid U U_3  \\
                  U_1 & \to BU                           \\
                  U_2 & \to BV                           \\
                  U_3 & \to BV                           \\
                  V   & \to AV_1 \mid B V_3 \mid \lambda \\
                  V_1 & \to VV_2                         \\
                  V_2 & \to BV                           \\
                  V_3 & \to VV_4                         \\
                  V_4 & \to AV                           \\
                  A   & \to a                            \\
                  B   & \to b
              \end{align*}

        \item Eliminăm producțiile cu \(\lambda\):
              \begin{align*}
                  S   & \to T \mid U                                      \\
                  T   & \to V T_1 \mid T_1 \mid V T_2 \mid T_2 \mid T T_3 \\
                  T_1 & \to AT                                            \\
                  T_2 & \to AV \mid A                                     \\
                  T_3 & \to AV \mid A                                     \\
                  U   & \to V U_1 \mid U_1 \mid V U_2 \mid U_2 \mid U U_3 \\
                  U_1 & \to BU                                            \\
                  U_2 & \to BV \mid B                                     \\
                  U_3 & \to BV \mid B                                     \\
                  V   & \to AV_1 \mid B V_3                               \\
                  V_1 & \to VV_2 \mid V_2                                 \\
                  V_2 & \to BV \mid B                                     \\
                  V_3 & \to VV_4 \mid V_4                                 \\
                  V_4 & \to AV \mid A                                     \\
                  A   & \to a                                             \\
                  B   & \to b
              \end{align*}

        \item Eliminăm producțiile unitate:
              \begin{align*}
                  S   & \to V T_1 \mid AT \mid V T_2 \mid AV \mid a \mid T T_3 \\
                  S   & \to V U_1 \mid BU \mid V U_2 \mid BV \mid b \mid U U_3 \\
                  T   & \to V T_1 \mid AT \mid V T_2 \mid AV \mid a \mid T T_3 \\
                  T_1 & \to AT                                                 \\
                  T_2 & \to AV \mid a                                          \\
                  T_3 & \to AV \mid a                                          \\
                  U   & \to V U_1 \mid BU \mid V U_2 \mid BV \mid b \mid U U_3 \\
                  U_1 & \to BU                                                 \\
                  U_2 & \to BV \mid b                                          \\
                  U_3 & \to BV \mid b                                          \\
                  V   & \to A V_1 \mid B V_3                                   \\
                  V_1 & \to VV_2 \mid BV \mid b                                \\
                  V_2 & \to BV \mid b                                          \\
                  V_3 & \to VV_4 \mid AV \mid a                                \\
                  V_4 & \to AV \mid a                                          \\
                  A   & \to a                                                  \\
                  B   & \to b
              \end{align*}
              Toate producțiile au fie doi neterminali în dreapta fie un singur terminal.
    \end{enumerate}
\end{proof}


\end{document}
