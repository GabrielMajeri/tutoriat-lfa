\documentclass[a4paper, 12pt]{article}

\usepackage{fontspec}
\usepackage{polyglossia}
\setdefaultlanguage{romanian}

\usepackage{mathtools}
\usepackage{amsthm}
\usepackage{braket}
\usepackage{unicode-math}

\newtheorem*{theorem}{Teoremă}

\theoremstyle{definition}
\newtheorem{definition}{Definiție}

\theoremstyle{remark}
\newtheorem*{example}{Exemplu}
\newtheorem*{property}{Proprietate}

\title{Minimizare DFA}
\author{To do}
\date{}

\DeclareMathOperator{\rhoequiv}{\rho}
\newcommand{\wordequiv}{\operatorname{\equiv}_{L}}
\newcommand{\notwordequiv}{\operatorname{\not\equiv}_{L}}
\newcommand{\dfaequiv}[1][M]{\operatorname{\equiv}_{#1}}

\begin{document}

\maketitle

Reamintim următoarele noțiuni de algebră:
\begin{itemize}
    \item Numim \emph{relație} (binară) peste o mulțime \(A\) orice submulțime \(\rho\) a produsului cartezian \(A \times A\).
    \item \(x \rhoequiv y\) se citește „\(x\) este în relație cu \(y\)” și înseamnă că perechea \((x, y) \in \rho\).
    \item Relația \(\rho\) este de \emph{echivalență} dacă este:
    \begin{itemize}
        \item \emph{reflexivă}: \(x \rhoequiv x\) pentru orice \(x \in A\)
        \item \emph{tranzitivă}: \(x \rhoequiv y \land y \rhoequiv z \implies x \rhoequiv z\), pentru orice \(x, y, z \in A\)
        \item \emph{simetrică}: \(x \rhoequiv y \implies y \rhoequiv x\), pentru orice \(x, y \in A\)
    \end{itemize}
    \item Notăm clasa de echivalență a lui \(x \in A\) cu \([x] = \Set{ a \in A | a \rhoequiv x}\)
    \item Mulțimea factor este mulțimea formată din toate clasele de echivalență  \(A / \rho = \Set{ [a] | a \in A }\).
    \item Spunem că o relație de echivalență este de indice finit dacă numărul de clase de echivalență determinate de aceasta este finit.
\end{itemize}

\begin{definition}[Relația de echivalență dată de un limbaj]\label{wordequiv}
Fie \(L \subseteq \Sigma^*\) un limbaj peste alfabetul \(\Sigma\). Definim relația binară pe cuvinte \(x \wordequiv y\) dacă și numai dacă \(\forall z \in \Sigma^*, xz \in L \iff yz \in L\).
Altfel spus, dacă cuvintele sunt echivalente, atunci oricum am adăuga un sufix la ele, nu le putem distinge.
\end{definition}

\begin{property}
Relația \ref{wordequiv} este de echivalență.
\end{property}
\begin{proof}
Reiese din faptul că \(\iff\) este relație de echivalență.
\end{proof}

\begin{example}
Pentru \(L = \Set{ a^i b^j | \forall i, j \geq 1 }\) și cuvintele \(x = a\) și \(y = aaa\), avem \(x \wordequiv y\). Dacă mai adăugăm orice sufix de forma \(z = a^m b^n\) ambele rămân în limbaj, iar orice alt sufix le va scoate pe ambele din limbaj.
\end{example}

\begin{definition}[Invarianță la dreapta la concatenare]
O relație binară pe cuvinte se numește \emph{invariantă la dreapta} față de concatenare dacă pentru orice două cuvinte \(x, y\) peste alfabetul \(\Sigma\), dacă cuvintele sunt în relație, atunci orice sufix le-am adăuga, ele rămân în relație.
\[
x \operatorname{R} y \implies \forall z \in \Sigma^*, xz \operatorname{R} yz
\]
\end{definition}

\begin{definition}[Relația dată de un automat]\label{dfaequiv}
Fie \(M = (Q, \Sigma, \delta, q_0, F)\) un DFA. Spunem că două cuvinte sunt echivalente față de \(M\) dacă plecând din starea inițială, cuvintele sunt ajung în aceeași stare.
\[
x \dfaequiv y \iff \delta(q_0, x) = \delta(q_0, y)
\]
\end{definition}

\begin{property}
Relația \ref{dfaequiv} este de echivalență.
\end{property}
\begin{proof}
Reiese din faptul că egalitatea (dintre stări, care sunt reprezentate prin numere) este o relație de echivalență.
\end{proof}

\begin{property}
Relația \ref{dfaequiv} este invariantă la dreapta.
\end{property}
\begin{proof}
Fie \(x, y \in \Sigma^*\) cu \(x \dfaequiv y\). Fie \(z \in \Sigma^*\) un cuvânt arbitrar. Trebuie să arătăm că \(xz \dfaequiv yz\):
\begin{align*}
    \delta(q_0, xz) &= \delta(\delta(q_0, x), z) & \text{din definiția lui } \delta \\
    &= \delta(\delta(q_0, y), z) & \text{din } x \dfaequiv y \\
    &= \delta(q_0, yz)
\end{align*}
\end{proof}

\begin{theorem}[Myhill--Nerode]
Fie \(L\) un limbaj peste \(\Sigma\). Următoarele afirmații sunt echivalente:
\begin{enumerate}
    \item \(L\) este regulat.
    \item \(L\) este reuniunea claselor de echivalență ale unei relații de echivalență invariantă la dreapta de indice finit.
    \item Relația de echivalență pe cuvinte \(\wordequiv\) este de indice finit.
\end{enumerate}
\end{theorem}
\begin{proof}
~
\begin{itemize}
    \item[\(1 \implies 2\)] Trebuie să construim o relație de echivalență invariantă la dreapta plecând de la ipoteza că limbajul este regulat.
    Fie \(M = (Q, \Sigma, \delta, q_0, F)\) un DFA care acceptă limbajul \(L\). Din acesta construim \(M'\), eliminând stările inaccesibile și completând automatul, care în continuare acceptă limbajul \(L\).
    Considerăm \(\dfaequiv[M']\), pentru care am demonstrat că este relație de echivalență și invariantă la dreapta.
    Vrem să demonstră că reuniunea claselor de echivalență asociate lui \(\dfaequiv[M']\) .
    \item[\(2 \implies 3\)]
    \item[\(3 \implies 1\)]
\end{itemize}
\end{proof}

\end{document}
