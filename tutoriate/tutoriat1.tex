\section*{Limbaje formale}

Orice problemă de decizie (adică cu răspuns „da” sau „nu”) din informatică poate fi redusă la determinarea apartenenței la un limbaj.
De exemplu, verificarea că un număr \(k\) este \emph{prim}, se poate face verificând dacă cuvântul \(\underbrace{aaa \dots aaa}_{k \text{ ori}}\) aparține limbajului \(\Set{ a^p | p \text{ prim} }\).

\subsection*{Limbaje regulate}

\textbf{Cum le recunoaștem:} Principala limitare a \dfa/\nfa/\lnfa{} este că nu au memorie. În formula pentru limbajele regulate pot apărea doar condiții liniare, de forma \(a^{nk + m}\), și dacă apar mai mulți indici la putere, aceștia nu sunt corelați.
s
\begin{figure}[H]
    \centering
    \begin{tikzpicture}
        \node[state, initial] (1) {\(q_1\)};
        \node[state, accepting, right of=1] (2) {\(q_2\)};

        \draw (1) edge[bend left, above] node{a} (2)
        (2) edge[bend left, below] node{a} (1);
    \end{tikzpicture}
    \caption*{Un DFA care acceptă limbajul \(\limbaj = \Set{a^{2k + 1} | k \in \naturals}\)}
\end{figure}

\begin{observation}
    Orice limbaj finit este regulat. Putem construi un DFA care să aibă câte o stare finală pentru fiecare cuvânt. Astfel se obține un \href{https://en.wikipedia.org/wiki/Trie}{trie}.
\end{observation}

\textbf{Proprietăți:} închise la intersecție, reuniune, complement, diferență de mulțime. Sunt închise și la concatenare și la stelare, și la morfisme și morfisme inverse.

\subsection*{Limbaje independente de context}

\textbf{Cum le recunoaștem:} Putem avea indici corelați (de exemplu \(a^{2k} b^{3k}\)).

Dacă apar mai mulți indici corelați, aceștia ar trebui să se grupeze asemenea parantezelor corect închise. De exemplu, se poate arăta că \(a^n b^m c^m d^n\) este independent de context dar \(a^n b^m c^n d^m\) \textbf{nu} este.

De asemenea, nu pot fi mai mult de două variabile corelate. De exemplu \(a^n b^n c^n\) este exemplul clasic de limbaj care \textbf{nu} este independent de context.

\textbf{Proprietăți:} închise la toate operațiile menționate mai sus \textbf{cu excepția} intersecție, complement, sau diferență. Sunt închise totuși la intersecția \emph{cu un limbaj regulat} (acest lucru ne ajută în exerciții).

\subsection*{Exerciții}

\begin{exercise}
    Scrieți un automat finit determinist care să accepte limbajul
    \[\limbaj = \Set{ a^{2k} | k \geq 0 }\]
\end{exercise}
\begin{proof}
    Cel mai simplu automat care acceptă acest limbaj este:
    \begin{figure}[H]
        \centering
        \begin{tikzpicture}
            \node[state, accepting, initial] (1) {\(q_1\)};
            \node[state, right of=1] (2) {\(q_2\)};

            \draw (1) edge[bend left, above] node{a} (2)
            (2) edge[bend left, below] node{a} (1);
        \end{tikzpicture}
    \end{figure}
\end{proof}

\begin{exercise}[exercițiul 7 din examen iunie 2011]
    Dați exemplu de un \nfa care nu este nici \lnfa, nici \dfa. Automatul trebuie să aibe minim 6 stări accesibile din starea inițială. Transformați automatul într-un \dfa.
\end{exercise}
\begin{proof}
    Ni se cere ca stările să fie accesibile, nu neapărat utile. Putem să modificăm automatul de mai sus:
    \begin{figure}[H]
        \centering
        \begin{tikzpicture}
            \node[state, accepting, initial] (1) {\(q_1\)};
            \node[state, right of=1] (2) {\(q_2\)};
            \node[state, above right of=1] (3) {\(q_3\)};
            \node[state, above left of=1] (4) {\(q_4\)};
            \node[state, below right of=1] (5) {\(q_5\)};
            \node[state, below left of=1] (6) {\(q_6\)};

            \draw (1) edge[bend left, above] node{a} (2)
            (2) edge[bend left, below] node{a} (1)
            (1) edge[bend left, above] node{a} (3)
            (1) edge[bend right, right] node{a} (4)
            (1) edge[bend right, below] node{a} (5)
            (1) edge[bend left, right] node{a} (6);
        \end{tikzpicture}
    \end{figure}

    Folosim metoda tabelului pentru a obține \dfa-ul echivalent:
    \begin{center}
        \begin{tabular}{c|c}
                                                   & \(a\)                     \\
            \hline
            Notăm cu \(A = \Set{q_1}\)             & \(\Set{q_2, \dots, q_6}\) \\
            \hline
            Notăm cu \(B = \Set{q_2, \dots, q_6}\) & \(\Set{q_1}\)
        \end{tabular}
    \end{center}
    Obținem DFA-ul inițial, unde \(q_1\) este \(A\) și \(q_2\) este \(B\).
\end{proof}

\begin{exercise}[bazat pe exercițiul 9 din examen iunie 2011]
    Construiți o gramatică independentă de context care să genereze limbajul
    \[\limbaj = \Set{ a^{2k} b^l a^k | k \geq 0, l \geq 1 }\]
\end{exercise}
\begin{proof}
    Vom genera partea exterioară a cuvântului (\textit{a}-urile) prin \(A\) și oricâte \textit{b}-uri prin \(B\).
    \begin{align*}
        S & \rightarrow A    \\
        A & \rightarrow aaAa \\
        A & \rightarrow B    \\
        B & \rightarrow bB   \\
        B & \rightarrow b
    \end{align*}
\end{proof}

\begin{exercise}[exercițiul 8 din examen iunie 2013]
    Construiți o gramatică independentă de context care să genereze limbajul
    \[\limbaj = \Set{ 0^{4k} 1^l 0^k | k \geq 0, l \geq 1 } \cdot \Set{ 0^i 1^{j + 3} | i \neq j}\]
\end{exercise}
\begin{proof}
    Pentru a genera prima mulțime, putem folosi producțiile:
    \begin{align*}
        S_1 & \rightarrow 0^4 S 0 \mid A \\
        A   & \rightarrow 1 A \mid 1
    \end{align*}
    Pentru a genera un număr diferit de \(0\) și de \(1\), trebuie să generăm unul din două cazuri: \(i > j\) sau \(i < j\).
    \begin{align*}
        S_2 & \rightarrow U \mid V                     \\
        U   & \rightarrow 0 U \mid 0 T \tag{\(i > j\)} \\
        V   & \rightarrow 1 V \mid 1 T \tag{\(i < j\)} \\
        T   & \rightarrow 0 T 1 \mid \lambda
    \end{align*}
    Le putem unii prin producția:
    \[S \rightarrow S_1 S_2 1^3\]
\end{proof}

\begin{exercise}[exercițiul 10 din examen iunie 2018]
    Construiți o gramatică independentă de context care să genereze limbajul
    \[\limbaj = \Set{ a^{m + n} b^k a^{m + k + i} b^n | i, k, m, n \geq 1 }\]
\end{exercise}
\begin{proof}
    Putem rescrie limbajul ca:
    \[\limbaj = \Set{ a^n a^m b^k a^k a^m a^i b^n | i, k, m, n \geq 1 }\]
    Acum se vede mult mai ușor că se poate genera limbajul:
    \begin{align*}
        S & \rightarrow a S b \mid a A b \tag{generează \(a^n \dots b^n\)} \\
        A & \rightarrow B C                                                \\
        B & \rightarrow a B a \mid a D a \tag{generează \(a^m \dots a^m\)} \\
        C & \rightarrow aC \mid a \tag{generează \(a^i\)}                  \\
        D & \rightarrow bDa \mid ba \tag{generează \(b^k a^k\)}
    \end{align*}
\end{proof}

\begin{exercise}[exercițiul 7, subpunctul \textit{a} din examen iunie 2016]
    Dați o gramatică independentă de context cu 7 producții, 2 din ele să fie producții unitare (unit production), și care are cel puțin 2 simboluri neterminale (nonterminating) și un simbol inaccesbil (unreachable).
\end{exercise}
\begin{proof}
    \begin{align*}
        S \rightarrow AB \mid C \mid \lambda                                  \\
        A \rightarrow a, B \rightarrow b \tag{\(A, B\) simboluri neterminale} \\
        C \rightarrow A, C \rightarrow B \tag{producții unitare}              \\
        X \rightarrow Y \tag{două simboluri inaccesibile}
    \end{align*}
\end{proof}

\begin{exercise}[exercițiul 7 din examen iunie 2017]
    Pentru următorul automat dați fiecare pas din algoritmul de construire al expresiei regulate echivalente:
    \begin{figure}[H]
        \centering
        \begin{tikzpicture}
            \node[state, initial] (1) {\(q_1\)};
            \node[state, accepting, right of=1] (2) {\(q_2\)};
            \node[state, right of=2] (3) {\(q_3\)};
            \node[state, accepting, below right of=2] (4) {\(q_4\)};

            \draw (1) edge[loop above] node{a} (1)
            (1) edge[above] node{b} (2)
            (2) edge[loop above] node{b} (2)
            (2) edge[bend left, above] node{a} (3)
            (2) edge[bend right, below] node{a} (4)
            (3) edge[loop right] node{c} (3)
            (3) edge[bend left, right] node{b} (4)
            (4) edge[bend right, above] node{a} (2);
        \end{tikzpicture}
    \end{figure}
\end{exercise}
\begin{proof}
    Realizăm o serie de transformări plecând de la automatul inițial.
    \begin{figure}[H]
        \centering
        \caption*{Adăugăm o nouă stare inițială și modificăm automatul să aibă o singură stare finală:}
        \begin{tikzpicture}
            \node[state, initial] (0) {\(q_0\)};
            \node[state, right of=0] (1) {\(q_1\)};
            \node[state, right of=1] (2) {\(q_2\)};
            \node[state, right of=2] (3) {\(q_3\)};
            \node[state, below right of=2] (4) {\(q_4\)};
            \node[state, accepting, below of=2] (5) {\(q_5\)};

            \draw (0) edge[above] node{\(\lambda\)} (1)
            (1) edge[loop above] node{a} (1)
            (1) edge[above] node{b} (2)
            (2) edge[loop above] node{b} (2)
            (2) edge[bend left, above] node{a} (3)
            (2) edge[bend right, below] node{a} (4)
            (2) edge[bend right, left] node{\(\lambda\)} (5)
            (3) edge[loop right] node{c} (3)
            (3) edge[bend left, right] node{b} (4)
            (4) edge[bend right, above] node{a} (2)
            (4) edge[bend left, below] node{\(\lambda\)} (5);
        \end{tikzpicture}
    \end{figure}
    \begin{figure}[H]
        \centering
        \caption*{Începem să eliminăm din stări. Începem cu \(q_1\):}
        \begin{tikzpicture}
            \node[state, initial] (0) {\(q_0\)};
            \node[state, right of=0] (2) {\(q_2\)};
            \node[state, right of=2] (3) {\(q_3\)};
            \node[state, below right of=2] (4) {\(q_4\)};
            \node[state, accepting, below of=2] (5) {\(q_5\)};

            \draw (0) edge[above] node{\(a^* b\)} (2)
            (2) edge[loop above] node{b} (2)
            (2) edge[bend left, above] node{a} (3)
            (2) edge[bend right, below] node{a} (4)
            (2) edge[bend right, left] node{\(\lambda\)} (5)
            (3) edge[loop right] node{c} (3)
            (3) edge[bend left, right] node{b} (4)
            (4) edge[bend right, above] node{a} (2)
            (4) edge[bend left, below] node{\(\lambda\)} (5);
        \end{tikzpicture}
    \end{figure}
    \begin{figure}[H]
        \centering
        \caption*{Eliminăm \(q_3\):}
        \begin{tikzpicture}
            \node[state, initial] (0) {\(q_0\)};
            \node[state, right of=0] (2) {\(q_2\)};
            \node[state, below right of=2] (4) {\(q_4\)};
            \node[state, accepting, below of=2] (5) {\(q_5\)};

            \draw (0) edge[above] node{\(a^* b\)} (2)
            (2) edge[loop above] node{b} (2)
            (2) edge[bend left=75, right] node{\(a c^* b\)} (4)
            (2) edge[bend right, below] node{a} (4)
            (2) edge[bend right, left] node{\(\lambda\)} (5)
            (4) edge[bend right, above] node{a} (2)
            (4) edge[bend left, below] node{\(\lambda\)} (5);
        \end{tikzpicture}
    \end{figure}
    \begin{figure}[H]
        \centering
        \caption*{Reunim muchiile de la \(q_2\) la \(q_4\):}
        \begin{tikzpicture}
            \node[state, initial] (0) {\(q_0\)};
            \node[state, right of=0] (2) {\(q_2\)};
            \node[state, right of=2] (4) {\(q_4\)};
            \node[state, accepting, below right of=2] (5) {\(q_5\)};

            \draw (0) edge[above] node{\(a^* b\)} (2)
            (2) edge[loop above] node{b} (2)
            (2) edge[bend left, above] node{\(a + a c^* b\)} (4)
            (2) edge[bend right, left] node{\(\lambda\)} (5)
            (4) edge[bend left, below] node{a} (2)
            (4) edge[bend left, right] node{\(\lambda\)} (5);
        \end{tikzpicture}
    \end{figure}
    \begin{figure}[H]
        \centering
        \caption*{Eliminăm starea \(q_4\):}
        \begin{tikzpicture}
            \node[state, initial] (0) {\(q_0\)};
            \node[state, right of=0] (2) {\(q_2\)};
            \node[state, accepting, below right of=2] (5) {\(q_5\)};

            \draw (0) edge[above] node{\(a^* b\)} (2)
            (2) edge[loop above] node{\(b + (b^* (a + ac^* b) a)^*\)} (2)
            (2) edge[bend left, right] node{\((a + ac^* b) \cdot (a b^* (a + ac^* b))^*\)} (5)
            (2) edge[bend right, left] node{\(\lambda\)} (5);
        \end{tikzpicture}
    \end{figure}
    \begin{figure}[H]
        \centering
        \caption*{Reunim muchiile de la \(q_2\) la \(q_5\):}
        \begin{tikzpicture}
            \node[state, initial] (0) {\(q_0\)};
            \node[state, right of=0] (2) {\(q_2\)};
            \node[state, accepting, below right of=2] (5) {\(q_5\)};

            \draw (0) edge[above] node{\(a^* b\)} (2)
            (2) edge[loop above] node{\(b + (b^* (a + ac^* b) a)^*\)} (2)
            (2) edge[bend left, right] node{\(\lambda + (a + ac^* b) \cdot (a b^* (a + ac^* b))^*\)} (5);
        \end{tikzpicture}
    \end{figure}

    \begin{figure}[H]
        \centering
        \caption*{Eliminăm starea \(q_2\):}
        \begin{tikzpicture}
            \node[state, initial] (0) {\(q_0\)};
            \node[state, accepting, right of=0] (5) {\(q_5\)};

            \draw (0) edge[below] node{\(E\)} (5);
        \end{tikzpicture}
    \end{figure}

    Unde \(E\) este expresia regulată cerută:
    \[a^* b (b + (b^* (a + ac^* b) a)^*)^* \cdot (\lambda + (a + ac^* b) \cdot (a b^* (a + ac^* b))^*)\]
\end{proof}
