\section*{Lema de pompare}

Să se demonstreze că limbajul
\[\limbaj = \Set{ a^n b^n c^2 | n \in \naturals^* }\]
nu este regulat.

Pentru a demonstra că acesta nu este un limbaj regulat, trebuie să demonstrăm că nu există niciun \dfa/\nfa/\lnfa{} care să îl accepte.

Când vrei să demonstrezi că ceva nu este posibil în matematică, de obicei folosești \emph{reducerea la absurd}. Presupui că o propoziție este adevărată, și ajungi la o contradicție.

O demonstrație că un limbaj nu este regulat/nu este independent de context decurge în felul următor:
\begin{enumerate}
    \item Presupun că limbajul este regulat/independent de context.

    \item \textbf{Dacă mă ajută,} pot să aplic orice operații la care este închis (deoarece am presupus că limbajul inițial este regulat/independent de context, în urma acestor operații obțin tot un limbaj regulat/independent de context):
          \begin{itemize}
              \item pentru limbaje regulate: cam toate operațiile
              \item pentru limbaje independente de context: intersecție cu limbaje regulate, morfisme și morfisme inverse
          \end{itemize}

    \item Pe rezultat aplic lema de pompare.
          Caut un contra-exemplu de cuvânt și \(i\) pentru care lema este falsă, și astfel am o contradicție

    \item În concluzie, limbajul nu este regulat/independent de context.
\end{enumerate}

\begin{observation}
    Lemele de pompare pot fi folosite doar pentru a demonstra că un limbaj \textbf{nu} este regulat/independent de context. Pentru a demonstra afirmativa trebuie să construim automatul/gramatica corespunzătoare.
\end{observation}

\subsection*{Pentru limbaje regulate}

Fie \(\limbaj\) un limbaj regulat. Există un \(n_0 \in \naturals^*\) care depinde de limbaj, cu proprietatea că orice cuvânt \(w \in \limbaj\), cu \(\abs{w} \geq n_0\), se poate descompune în \(w = x y z\) cu proprietățile:
\begin{itemize}
    \item \(\abs{y} \geq 1\)
    \item \(\abs{xy} \leq n_0\)
    \item \(\forall i \in \naturals\), \(x y^i z \in \limbaj\)
\end{itemize}

\subsection*{Pentru limbaje independente de context}

Fie \(\limbaj\) un limbaj independent de context. Există un \(n_0 \in \naturals^*\) care depinde de limbaj, cu proprietatea că orice cuvânt \(w \in \limbaj\), cu \(\abs{w} \geq n_0\), se poate descompune în \(w = x u y v z\) cu proprietățile:
\begin{itemize}
    \item \(\abs{u v} \geq 1\)
    \item \(\abs{u y v} \leq n_0\)
    \item \(\forall i \in \naturals\), \(x u^i y v^i z \in \limbaj\)
\end{itemize}

\subsection*{Exerciții}

\begin{exercise}[exercițiul 8 din examen iunie 2011]
    Demonstrați folosind lema de pompare că limbajul
    \[\limbaj = \set{ a^{2k} b^l a^{k} | k \geq 0, l \geq 1 }\]
    nu este regulat.
\end{exercise}
\begin{proof}
    Partea din limbaj care „ne deranjează” este cea cu \(k\), unde avem o corelare a numărului de \(a\)-uri de la început cu numărul de \(a\)-uri de la sfârșit.

    Ca să simplificăm problema, vom intersecta limbajul cu un limbaj regulat.

    Presupunem că \(\limbaj\) este regulat. Atunci limbajul
    \[\limbaj' = \limbaj \cap a^* b a^* = \set{ a^{2k} b a^k | k \geq 0 }\]
    este la rândul lui regulat (am folosit o expresie regulată ca să descriu limbajul cu care intersectez).

    Fie \(n_0 \in \naturals^*\) constanta din lema de pompare pentru limbaje regulate.

    Să luăm cuvântul
    \[w = a^{2 n_0} \, b \, a^{n_0}\]
    Acesta are lungimea \(\abs{w} = 2 n_0 + 1 + n_0 = 3n_0 + 1 \geq n_0\), deci putem aplica lema pe el.

    Trebuie să ne gândim cum ar arăta o descompunere a cuvântului care să respecte proprietățile din lemă:
    \begin{align*}
        w        & = x \, y \, z \\
        \abs{y}  & \geq 1        \\
        \abs{xy} & \leq n_0
    \end{align*}

    Datorită modului în care am ales cuvântul, o descompunere care convine o să fie de forma
    \[
        w = \underbrace{aa \dots aa}_{x}
        \underbrace{aa \dots aa}_{y}
        \underbrace{aa \dots aa b aa \dots aa}_{z}
    \]
    Scris pe scurt:
    \begin{gather*}
        x = a^p \\
        y = a^q \\
        z = a^{2 n_0 - q - p} \, b \, a^{n_0}
    \end{gather*}
    unde
    \begin{align*}
        q     & \geq 1   \\
        p + q & \leq n_0
    \end{align*}

    Din lemă avem că
    \[\forall i \in \naturals, x y^i z \in \limbaj'\]
    Pentru \(i = 0\) obținem
    \begin{align*}
        x y^0 z & = a^p \, a^{2 n_0 - q - p} \, b \, a^{n_0} = \\
                & = a^{2 n_0 - q} \, b \, a^{n_0}
    \end{align*}
    Dacă aparține limbajului, ar rezulta că
    \[2 n_0 - q = 2 n_0\]
    Dar știm deja că \(q \geq 1\), deci contradicție.

    Presupunerea că \(\limbaj\) este regulat este falsă.
\end{proof}

\begin{exercise}[exercițiul 8 din examen iunie 2014]
    Demonstrați că limbajul
    \[\limbaj = \set{ a^k b^{3l} a^l | k \geq 1, l \geq 0 }\]
    nu este regulat.
\end{exercise}
\begin{proof}
    Presupunem că \(\limbaj\) este regulat.

    Putem intersecta limbajul cu \(a b^* a^*\), și obținem
    \[\limbaj' = \Set{ a b^{3l} a^l | l \geq 0 }\]

    Fie \(n_0 \in \naturals\) din lema de pompare pentru limbaje regulate.

    Putem lua cuvântul
    \[w = a \, b^{3 n_0} \, a^{n_0}\]
    care are lungimea \(\abs{w} = 1 + 3 n_0 + n_0 = 4 n_0 + 1 \geq n_0\).

    Trebuie să luăm în considerare toate descompunerile posibile pentru acest cuvânt.

    Să ne gândim cum arată cuvântul \(w\):

    \[w = \underbrace{a}_{\text{un \(a\)}} \underbrace{bbbbbbb\dots{}bbbb}_{3n_0 \text{ de \(b\)-uri}} \, \underbrace{aaa\dots{}aaa}_{n_0 \text{ de \(a\)-uri}}\]

    Știm că \(xy\) trebuie să se afle la începutul cuvântului, și din lema de pompare avem \(\abs{xy} \leq n_0\).

    \begin{enumerate}
        \item Cazul în care \(x\) îl conține pe \(a\):
              \[w = \underbrace{a bb\dots{}bbb}_{x} \underbrace{b\dots{}bb}_{y} \underbrace{bbb\dots{}bbb aa\dots{}aa}_{z}\]
              \begin{gather*}
                  x = a \, b^p \\
                  y = b^q \\
                  z = b^{3 n_0 - q - p} a^{n_0}
              \end{gather*}
              unde
              \begin{align*}
                  \abs{y}   & \geq 1 \implies q \geq 1 \\
                  \abs{x y} & \leq n_0
              \end{align*}

              Din lema de pompare avem că
              \[\forall i \in \naturals, x \, y^i \, z \in \limbaj'\]

              Luăm \(i = 0\). Obținem cuvântul
              \begin{align*}
                  x y^0 z & = a \, b^p \, b^{3 n_0 - q - p} \, a^{n_0} \\
                          & = a \, b^{3 n_0 - q} \, a^{n_0}
              \end{align*}
              care nu aparține limbajului deoarece \(3 n_0 - q < 3 n_0\).

              Contradicție cu lema de pompare, deci \(\limbaj'\) nu este regulat.

        \item Cazul în care \(x\) nu-l conține pe \(a\) (este \(\lambda\)):
              \[w = \underbrace{}_{x} \underbrace{a bb\dots{}bbb}_{y} \underbrace{bbb\dots{}bbb aa\dots{}aa}_{z}\]
              \begin{gather*}
                  x = \lambda \\
                  y = a \, b^p \\
                  z = b^{3 n_0 - p} \, a^{n_0}
              \end{gather*}
              unde
              \[1 \leq \abs{y} \leq n_0\]

              După un raționament analog ajungem la aceeași concluzie. În momentul în care ridicăm \(y\) la o putere \(i \geq 2\), o să avem mai mult de un \(a\) la început.
    \end{enumerate}

\end{proof}

\begin{exercise}[exercițiul 9 din examen iunie 2016]
    Demonstrați că limbajul
    \[\limbaj = \set{ w a^k w | w \in \set{a, b, c}^*, k \geq 0 }\]
    nu este regulat.
\end{exercise}
\begin{proof}
    Presupunem că \(\limbaj\) ar fi regulat.

    Fie \(n_0 \in \naturals\) din lema de pompare.

    Putem alege \(w = b^{n_0} \, a \, b^{n_0}\), cu \(\abs{w} = n_0 + 1 + n_0 = 2 n_0 + 1 \geq n_0\).

    Avem
    \begin{gather*}
        x = b^{p} \\
        y = b^{q} \\
        z = b^{n_0 - q - p} \, a \, b^{n_0}
    \end{gather*}
    cu
    \begin{align*}
        \abs{y}   & \geq 1   \\
        \abs{x y} & \leq n_0
    \end{align*}

    Aplicăm lema de pompare pentru \(i = 0\) și obținem
    \begin{align*}
        x y^0 z & = b^p \, b^{n_0 - q - p} \, a \, b^{n_0} \\
                & = b^{n_0 - q} \, a \, b^{n_0}
    \end{align*}
    Deoarece \(n_0 - q \neq n_0\), cuvântul nu aparține limbajului.

    Prin urmare, presupunerea noastră este falsă, limbajul nu este regulat.
\end{proof}

\begin{exercise}[exercițiul 10 din examen iunie 2017]
    Demonstrați că limbajul
    \[\limbaj = \set{ a^n b^m | n \text{ pătrat perfect}, m \geq 5 }\]
    nu este independent de context.
\end{exercise}
\begin{proof}
    Presupunem că \(\limbaj\) este independent de context.

    Îl intersectăm cu \(a^* b^5\) și obținem
    \[\limbaj' = \set{ a^n b^5 | n \text{ pătrat perfect } }\]

    Fie \(n_0 \in \naturals\) din lema de pompare.

    Alegem cuvântul
    \[w = a^{(n_0)^2} b^5\]
    care este din limbaj și are \(\abs{w} = (n_0)^2 + 5 \geq n_0\)

    Dacă luăm o descompunere în care \(u y v\) se află în zona de \(b\)-uri, atunci când o să aplicăm lema de pompare pentru \(i \neq 1\) nu mai avem \(b^5\).

    Dacă luăm o descompunere în care \(u y v\) conține și \(a\)-uri, de exemplu:
    \begin{gather*}
        x = a^p \\
        u = a^q \\
        y = a^r \\
        v = a^s \\
        z = a^{(n_0)^2 - s - r - q - p} \, b^5
    \end{gather*}
    unde
    \begin{align*}
        \abs{uv} \geq 1      & \iff q + s \geq 1       \\
        \abs{u y v} \leq n_0 & \iff q + r + s \leq n_0
    \end{align*}

    Aplicăm lema de pompare pentru un \(i\) oarecare și vedem ce obținem:
    \begin{align*}
        x u^i y v^i z & = a^p \, a^{iq} \, a^r \, a^{is} \, a^{(n_0)^2 - s - r - q - p} \, b^5 \\
                      & = a^{(n_0)^2 + q (i - 1) + s (i - 1)} \, b^5                           \\
                      & = a^{(n_0)^2 + (q + s) (i - 1)} \, b^5
    \end{align*}

    Între două pătrate perfecte nu poate exista un alt pătrat perfect. După \((n_0)^2\), următorul pătrat perfect este \((n_0 + 1)^2\), adică \((n_0)^2 + 2 n_0 + 1\).

    Dacă luăm \(i = 2\), obținem
    \[a^{(n_0)^2 + (q + s)} \, b^5\]

    Cu siguranță \(q + s \leq 2 n_0 + 1\), deci nu avem un pătrat perfect. Cuvântul nu aparține limbajului.

    În concluzie, \(\limbaj\) nu este independent de context.
\end{proof}

\begin{exercise}[exercițiul bonus din examen iunie 2013]
    Demonstrați că limbajul
    \[\limbaj = \set{ 0^n 1^m | n > 0, m \text{ prim} }\]
    nu este independent de context.
\end{exercise}
\begin{proof}
    Presupunem că \(\limbaj\) este independent de context.

    Îl putem intersecta cu \(0 1^*\) și obținem
    \[\limbaj = \set{ 0 1^m | m \text{ prim } }\]

    Fie \(n_0 \in \naturals\) din lema de pompare.

    Notăm cu \(P\) = următorul număr prim mai mare decât \(n_0\).
    Cuvântul \(w = 0 1^P\) aparține limbajului și are lungime \(\abs{w} = 1 + P \geq n_0\).

    Avem mai multe cazuri, în funcție de cum este descompunerea:
    \begin{enumerate}
        \item
              \begin{gather*}
                  x = 0 \, 1^p \\
                  u = 1^q \\
                  y = 1^r \\
                  v = 1^s \\
                  z = 1^{P - s - r - q - p}
              \end{gather*}
              unde
              \begin{align*}
                  \abs{u v} \geq 1     & \iff q + s \geq 1       \\
                  \abs{u y v} \leq n_0 & \iff q + r + s \leq n_0
              \end{align*}

              Aplicăm lema de pompare pentru un \(i \in \naturals\) oarecare:
              \begin{align*}
                  x u^i y v^i z & = 0 \, 1^p \, 1^{iq} \, 1^r \, 1^{is} \, 1^{P - s - r - q - p} \\
                                & = 0 \, 1^{P + q(i - 1) + s(i - 1)}                             \\
                                & = 0 \, 1^{P + (q + s)(i - 1)}
              \end{align*}

              Pentru \(i = P + 1\), obținem cuvântul
              \[0 \, 1^{P + (q + s) P} = 0 \, 1^{(q + s)(P + 1)}\]
              care nu aparține limbajului, deoarece exponentul lui \(1\) este număr compus. Contradicție cu lema de pompare.

        \item
              \begin{gather*}
                  x = \lambda \\
                  u = 0 \, 1^p \\
                  y = 1^q \\
                  v = 1^r \\
                  z = 1^{P - r - q - p}
              \end{gather*}

              Aplicând lema de pompare obținem un rezultat analog cu primul caz.

        \item
              \begin{gather*}
                  x = u = \lambda \\
                  y = 0 \, 1^p \\
                  v = 1^q \\
                  z = 1^{P - q - p}
              \end{gather*}

              Asemănător cu cazurile de mai sus.

        \item
              \begin{gather*}
                  x = u = y = \lambda \\
                  v = 0 \, 1^p \\
                  z = 1^{P - p}
              \end{gather*}

              Asemănător cu cazurile de mai sus.
    \end{enumerate}

    În concluzie, \(\limbaj\) nu este independent de context.
\end{proof}

\begin{exercise}[exercițiul 11 din examen iunie 2011]
    Demonstrați că limbajul
    \[\limbaj = \set{ a^i b^j c^k | i < j \text{ și } i + 2j + 3 < k }\]
    nu este independent de context.
\end{exercise}
\begin{proof}
    Presupunem că \(lang\) este independent de context.

    Fie \(n_0\) din lema de pompare.

    Vrem să alegem un cuvânt în care inegălitățile să fie satisfăcute cât mai la limită.
    Alegem \(i = n_0\), din \(i < j\) putem alege \(j = n_0 + 1\).
    Din \(i + 2j + 3 < k\), alegem \(k = 3 n_0 + 6\).

    Cuvântul ales este \(w = a^{n_0} \, b^{n_0 + 1} \, c^{3 n_0 + 6}\), unde \(\abs{w} = 5 n_0 + 7 \geq n_0\).

    Obținem foarte multe descompuneri posibile:

    Dacă
    \begin{gather*}
        x = a^p \\
        u = a^q \\
        y = a^r \\
        v = a^s \\
        z = a^{n_0 - s - r - q - p} \, b^{n_0 + 1} \, c^{3 n_0 + 6}
    \end{gather*}
    putem pompa cu \(i > 1\) și vom obține un număr de \(a\)-uri care depășește numărul de \(b\)-uri, ieșind din limbaj.

    Dacă
    \begin{gather*}
        x = a^{n_0} b^p \\
        u = b^q \\
        y = b^r \\
        v = b^s \\
        z = b^{n_0 + 1 - s - r - q - p} c^{3 n_0 + 6}
    \end{gather*}
    putem pompa cu \(i > 1\) și vom depăși a doua condiție.

    etc.
\end{proof}

\begin{exercise}[exercițiul 10 din examen iunie 2016]
    Demonstrați că limbajul
    \[\limbaj = \set{ a^n b^m c^r | n \geq m \geq r \geq 150 }\]
    nu este independent de context.
\end{exercise}
\begin{proof}
    Limbajul se poate rescrie ca
    \[\limbaj = \set{ a^n b^m c^r | n \geq m, m \geq r, r \geq 150 }\]

    Alegem un cuvânt cum ar fi \(w = a^{n_0} b^{n_0} c^{n_0}\) și aplicăm lema de pompare.

    \begin{itemize}
        \item În cazurile în care pompăm doar \(a\)-uri putem lua \(i = 0\) și o să avem mai puține \(a\)-uri decât \(b\)-uri.
        \item Când pompăm și \(a\)-uri și \(b\)-uri sau doar \(b\)-uri putem lua \(i = 0\) și avem mai puține \(b\)-uri decât \(c\)-uri.
        \item Când pompăm \(c\)-uri putem lua un \(i > 1\) și o să avem mai multe \(c\)-uri decât \(b\)-uri.
    \end{itemize}
\end{proof}
