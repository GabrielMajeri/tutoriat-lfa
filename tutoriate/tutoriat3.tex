\section*{Teorie la examen}

Prima jumătate a examenului se bazează pe teoria învățată la curs.

La primele două probleme se cere demonstrația a unor teoreme/propoziții din curs.

Urmează câte 4 afirmații a căror valoare de adevăr trebuie determinată, cu justificări (pe scurt și la obiect). Acestea nu apar explicit în curs, dar se bazează pe observațiile și informațiile din materiale.

\subsection*{Decidabilitate}

Ca să arăți că o anumită problemă este \emph{decidabilă}, trebuie să arăți că există un algoritm pentru asta. De exemplu, să verifici că un număr e prim este decidabil, pentru că îl poți descompune în factori și vezi să nu aibă alți factori în afară de el însuși.

Ca să arăți că o anumită problemă \emph{nu este decidabilă}, trebuie să te bazezi pe afirmații din curs. De exemplu, se știe că nu există un algoritm pentru a găsi intersecția a două gramatici independente de context.

\subsection*{Egalitatea de limbaje}

Ca să verifici dacă două limbaje regulate \(\lang_1, \lang_2\) sunt egale, construiești automate finite care să accepte \(\lang_1\) respectiv \(\lang_2\), apoi aplici algoritmul din curs pentru a verifica că automatele sunt echivalente.

Pentru limbaje independente de context, nu se poate verifica (nu este decidabilă) egalitatea.

\subsection*{Exerciții de teorie}

Pentru fiecare dintre următoarele exerciții, cerința este să se \textbf{justifice} dacă afirmația respectivă este \emph{adevărată} sau \emph{falsă}.

\subsubsection*{Examen 2011}

\begin{exercise}
    Există limbaje regulate care nu sunt independente de context.
\end{exercise}
\begin{proof}
    Afirmația este falsă, limbajele regulate sunt incluse strict în cele independente de context, conform \href{https://en.wikipedia.org/wiki/Chomsky_hierarchy}{ierarhiei Chomsky}.

    Pentru orice limbaj regulat putem scrie o gramatică regulată corespunzătoare, care este inclusă în gramaticile independente de context, deci și limbajul este inclus în limbajele independente de context.
\end{proof}

\begin{exercise}
    Fie limbajele \(\lang_1, \lang_2\) cu proprietatea că  \(\lang_1 \subseteq \lang_2\) și \(\lang_2 \in REG\). Atunci \(\lang_1 \in REG\).
\end{exercise}
\begin{proof}
    Fals, un contraexemplu este să luăm \(\lang_1 = \set{ a^n b^n | n \in \naturals }\), \(\lang_2 = \set{ a, b }^*\).

    Avem că \(\lang_1 \subseteq \lang_2\) dar știm că \(\lang_1\) nu este regulat.
\end{proof}

\begin{exercise}
    Este decidabil dacă limbajele acceptate de doua automate finite deterministe sunt egale sau nu.
\end{exercise}
\begin{proof}
    Dacă avem două automate finite deterministe, putem aplica algoritmul de minimizare din curs (cel bazat pe un tabel). În urma acestui algoritm putem determina dacă automatele sunt echivalente.
\end{proof}

\begin{exercise}
    Este decidabil dacă limbajul intersecției a două gramatici independente de context este vid sau nu.
\end{exercise}
\begin{proof}
    Nu există un algoritm pentru determinarea intersecției a două gramatici independente de context. De aceea, nu se poate verifica dacă aceasta este vidă sau nu.
\end{proof}

\subsubsection*{Examen 2013}

\begin{exercise}
    Există o gramatică regulată \(G\) astfel încât nu există niciun automat finit nedeterminist (\nfa) care să accepte exact \(\lang(G)\).
\end{exercise}
\begin{proof}
    Afirmația este falsă. Prin definiție, gramaticile \emph{regulate} definesc un limbaj regulat. Pentru orice limbaj regulat se poate scrie un \nfa{}.
\end{proof}

\begin{exercise}
    Este decidabil dacă limbajele acceptate de două automate finite nedeterministe cu lambda mișcări sunt egale sau nu.
\end{exercise}
\begin{proof}
    Orice \lnfa{} poate fi redus la un \dfa{}, deci această întrebare este echivalentă cu exercițiul 3.
\end{proof}

\begin{exercise}
    Fie limbajele \(\lang_1, \lang_2\) cu proprietatea că \(\lang_2 \subseteq \lang_1\) și \(\lang_2 \in CF\). Atunci \(\lang_1 \in CF\).
\end{exercise}
\begin{proof}
    Putem avea \(\lang_2 = \set{ a^n b^n | n \in \naturals }\) care știm că este independent de context, și \(\lang_1 = \lang_2 \cup \set{ c^p | p \text{ prim} }\).

    Avem că \(\lang_2 \subseteq \lang_1\) dar \(\lang_1 \not\in CF\).
\end{proof}

\begin{exercise}
    Există limbaje finite care nu sunt regulate.
\end{exercise}
\begin{proof}
    Dacă avem un limbaj finit format din cuvintele \(\set{ w_1, \dots, w_n }\), putem să scriem o expresie regulată \(w_1 + \dots + w_n\) care să îl accepte.
\end{proof}

\subsubsection*{Examen 2014}

\begin{exercise}
    Există o gramatică regulată \(G\) astfel încât nu există nicio expresie regulată \(E\) cu proprietatea că \(\lang(E) = \lang(G)\).
\end{exercise}
\begin{proof}
    La fel ca la exercițiul 5, gramaticile regulate și expresiile regulate definesc aceeași familie de limbaje, limbajele regulate.
\end{proof}

\begin{exercise}
    Fie limbajele \(\lang_1, \lang_2\) cu proprietatea că \(\lang_1 \subseteq \lang_2\) și \(\lang_1 \in REG\). Deci \(\lang_2 \in REG\).
\end{exercise}
\begin{proof}
    Fals, putem da un contraexemplu ca la exercițiul 7, cu \(\lang_1 = \set{ a^n | n  }\) și \(\lang_2 = \lang_1 \cup \set{ b^p | p \text{ prim}}\).
\end{proof}

\begin{exercise}
    Este decidabil dacă limbajele acceptate de două expresii regulate sunt egale.
\end{exercise}
\begin{proof}
    Expresiile regulate pot fi transformate în automate finite. Putem verifica echivalența automatelor finite ca la exercițiul 3.
\end{proof}

\begin{exercise}
    Există limbaje modelate de automate push-down deterministe care au toate cuvintele de lungime impară și nu pot fi modelate de gramatici independente de context.
\end{exercise}
\begin{proof}
    Nu există, limbajele acceptate de \emph{orice fel} de automat push-down sunt incluse în limbajele modelate de gramaticile independente de context.
\end{proof}

\subsubsection*{Examen 2016}

\begin{exercise}
    Fie limbajele \(\lang_1, \lang_2, \lang_3\) cu proprietatea că \(\lang_1 \cup \lang_2 = \lang_3\) și \(\lang_2, \lang_3 \in REG\). Deci \(\lang_1 \in REG\).
\end{exercise}
\begin{proof}
    Fals, putem avea \(\lang_2 = \lang_3 = \set{ a, b }^*\), și \(\lang_1 = \set{ a^n b^n | n \in \naturals}\), care nu este regulat.
\end{proof}

\begin{exercise}
    Există o gramatică regulată \(G\) peste alfabetul \(\set{ a, b, c }\) astfel încât nu există niciun \nfa{} \(A\) cu proprietatea că
    \[\lang(A) = \lang(G) \cup \set{ acccab, bbaabb }\]
\end{exercise}
\begin{proof}
    Limbajul modelat de \(G\) este unul regulat, deoarece gramatica este regulată. De asemenea \(\set{ acccab, bbaabb }\) este o mulțime finită de cuvinte, deci este un limbaj finit, deci limbaj regulat.

    Reuniunea a două limbaje regulate este tot un limbaj regulat. Deci termenul din dreapta este sigur acceptabil de un \nfa{}.
\end{proof}

\begin{exercise}
    Există limbaje modelate de gramatici independente de context care au toate cuvintele de lungime impară și nu pot fi modelate de automate push-down deterministe?
\end{exercise}
\begin{proof}
    Un exemplu de limbaj care nu poate fi acceptat de automate push-down \emph{deterministe} este \(\set{ w w^R | w \in \set{a, b}^* }\), unde \(R\) înseamnă oglinditul cuvântului.

    Toate cuvintele din acest limbaj au lungime \(\abs{w} + \abs{w^R} = 2 \abs{w}\), care este un număr par. Trebuie să mai adăugăm o literă ca lungimea să fie impară.

    Dacă adăugăm o literă distinctivă în mijloc (de exemplu un \(c\)), limbajul ar putea fi acceptat și de automate push-down deterministe. Soluția este să adăugăm litera la început sau la final:
    \[\lang = \set{ w w^R c | w \in \set{a, b}^* }\]
\end{proof}

\begin{exercise}
    Este decidabil dacă limbajele acceptate de un \nfa{} cu lambda mișcări și o gramatică regulată cu cel mult 20 de producții sunt egale.
\end{exercise}
\begin{proof}
    Orice gramatică regulată produce un limbaj regulat, și putem verifica echivalența de limbaje regulate ca la exercițiul 3.
\end{proof}

\subsubsection*{Examen 2017}

\begin{exercise}
    Fie limbajele \(\lang_1, \lang_2, \lang_3\) cu proprietatea că \(\lang_1 \cup \lang_2 = \lang_3\) și \(\lang_2, \lang_3 \in REG\). Deci \(\lang_1 \in REG\).
\end{exercise}
\begin{proof}
    Identic cu exercițiul 13.
\end{proof}

\begin{exercise}
    Există o gramatică neambiguă \(G\) peste alfabetul \(\set{ a, b, c }\) astfel încât nu există niciun \nfa{} \(A\) cu proprietatea că
    \[\lang(A) = \lang(G) \cup \set{ acccab, bbaabb }\]
\end{exercise}
\begin{proof}
    O gramatică \emph{neambiguă} este independentă de context, nu neapărat regulată. Putem să luăm contraexemplu gramatica
    \begin{align*}
        S & \rightarrow aSb     \\
        S & \rightarrow \lambda
    \end{align*}
    care este neambiguă și generează \(\set{ a^n b^n | n \in \naturals }\). Aceasta nu poate fi acceptată de un automat finit.
\end{proof}

\begin{exercise}
    Există limbaje modelate de gramatici independente de context care au toate cuvintele de lungime impară și nu pot fi modelate de automate push-down cu acceptare prin stare finală și stivă vidă.
\end{exercise}
\begin{proof}
    Fals, orice gramatică independentă de context poate fi modelată de un automat push-down. Dacă folosim automate push-down nedeterministe atunci modul de acceptare nu contează.
\end{proof}

\begin{exercise}
    Este decidabil dacă limbajele decidabile de o expresie regulată cu cel puțin 20 de operator și o gramatică regulată cu cel puțin 20 de producții sunt egale.
\end{exercise}
\begin{proof}
    Limbajele produse de expresiile regulate și de gramaticile regulate sunt limbaje regulate, deci putem să le verificăm echivalența ca la exercițiul 3.
\end{proof}

\subsubsection*{Examen 2018}

\begin{exercise}
    Este decidabil dacă limbajele acceptate de un \nfa{} cu lambda mișcări și un automat push-down aciclic sunt egale.
\end{exercise}
\begin{proof}
    Dacă automatul push-down este aciclic, atunci el nu poate produce o infinitate de cuvinte; dacă facem o mulțime cu toate cuvintele pe care le produce, aceasta va fi finită.

    Combinând asta cu concluzia de la exercițiul 8, care ne spune că orice limbaj (mulțime de cuvinte) finit este regulat, avem că limbajul acceptat de acest automat push-down este \emph{regulat}. Deci putem construi un automat finit care să îl accepte, și apoi să verificăm echivalența între acesta și un \nfa{}.
\end{proof}

\begin{exercise}
    Fie limbajele \(\lang_1, \lang_2, \lang_3\) cu proprietatea că \(\lang_1 \cap \lang_2 = \lang_3\) și \(\lang_1, \lang_2 \in CFL\). Deci \(\lang_3 \in CFL\).
\end{exercise}
\begin{proof}
    Fals, limbajele independente de context nu sunt închise la intersecție. Contraexemplul clasic este \(\lang_1 = \set{ a^n b^n c^i | n \in \naturals, i \in \naturals }\), \(\lang_2 = \set{ a^j b^m c^m | j \in \naturals, m \in \naturals }\).

    \(\lang_3\) nu este independent de context:
    \[\lang_1 \cap \lang_2 = \set{ a^n b^n c^n | n \in \naturals } = \lang_3\]
\end{proof}

\begin{exercise}
    Fie limbajele \(\lang_1, \lang_2, \lang_3\) cu proprietatea că \(\lang_1 \cap \overline{\lang_2} = \lang_3\) și \(\lang_2, \lang_3 \in REG\). Deci \(\lang_1 \in REG\).
\end{exercise}
\begin{proof}
    Fals, un contraexemplu este \(\lang_1 = \set{ a^n b^n | n \in \naturals }\), \(\lang_2 = \set{a, b}^*\), \(\lang_3 = \varnothing\).

    Avem că
    \[\lang_1 \cap \overline{\lang_2} = \lang_1 \cap \overline{\set{ a, b }^* } = \lang_1 \cap \varnothing = \varnothing = \lang_3\]
\end{proof}

\begin{exercise}
    Fie limbajele \(\lang_1, \lang_2, \lang_3\) cu proprietatea că \(\lang_1 \setminus \lang_2 = \lang_3\) și \(\lang_2, \lang_3 \in REG\). Deci \(\lang_1 \in REG\).
\end{exercise}
\begin{proof}
    Deoarece \(A \setminus B = A \cap \overline{B}\), această problemă este echivalentă cu exercițiul 23.
\end{proof}
