\section*{Automate Pushdown}

Un \emph{automat pushdown} este un automat finit extins cu o stivă de dimensiune infinită. Acesta operează la fel ca un automat obișnuit, prelucrând cuvintele literă cu literă, dar poate să stocheze și să ia decizii pe baza memoriei sale.

Limbajele acceptate de \pda{} sunt aceleași cu limbajele generate de \textbf{gramatici independente de context}.

Automatele pushdown deterministe sunt incluse \textbf{strict} în cele nedeterministe.
De exemplu, se poate demonstra că limbajul
\[\limbaj = \set{ w w^R | w \in \set{a, b}^* }\]
nu poate fi acceptat de un \dpda{}, dar este destul de simplu să facem un \npda{} pentru el.

Vom folosi simbolul \(Z_0\) pentru a indica o stivă vidă: în momentul în care \pda{}-ul începe execuția, presupunem că acest simbol se află pe stivă, și îl vom lăsa tot timpul la baza stivei.

\subsection*{Exerciții}

\begin{exercise}[exercițiul 10 din examen iunie 2011]
    Construiți un automat pushdown (\pda{}), eventual determinist, pentru limbajul
    \[
        \limbaj = \set{ w | w \in \set{a, b, c}^*, \abs{w}_a = \abs{w}_b + 2 }
    \]
\end{exercise}

\textit{Idee}: Pe stivă reținem dacă am citit mai multe \(a\)-uri sau mai multe \(b\)-uri până la acel moment. La finalul cuvântului, trebuie să avem două \(a\)-uri în plus.

\begin{proof}~
    \begin{figure}[H]
        \centering
        \caption*{Construim următorul \pda{}, cu acceptare prin stare finală și stivă vidă:}
        \begin{tikzpicture}
            \node[state, initial] (1) {\(q_1\)};
            \node[state, right of=1] (2) {\(q_2\)};
            \node[state, right of=2] (3) {\(q_3\)};
            \node[state, accepting, right of=3] (4) {\(q_4\)};

            \draw (1) edge[above] node[align=center] {
                    \(\lambda, Z_0/Z_0\) \\
                    \(\lambda, A/A\) \\
                    \(\lambda, B/B\)
                } (2)
            (1) edge[loop above] node[align=center] {
                    \(a, Z_0/AZ_0\) \\
                    \(a, A/AA\) \\
                    \(a, B/\lambda\) \\
                    \(b, Z_0/BZ_0\) \\
                    \(b, B/BB\) \\
                    \(b, A/\lambda\)
                } (1)
            (2) edge[above] node {\(\lambda, A/\lambda\)} (3)
            (3) edge[above] node {\(\lambda, A/\lambda\)} (4);
        \end{tikzpicture}
    \end{figure}

    În starea \(q_1\), automatul ciclează, citește caractere și le pune pe stivă. La un moment dat, acesta trece nedeterminist în \(q_2\). Tranzițiile de la \(q_2\) la \(q_3\) și de la \(q_3\) la \(q_4\) verifică că avem două \(a\)-uri în plus pe stivă.
\end{proof}

\begin{exercise}[exercițiul 11 din examen iunie 2013]
    Construiți un automat pushdown (\pda{}), eventual determinist, pentru limbajul
    \[
        \limbaj = \set{ w | w \in \set{0, 1, 2}^*, 4\abs{w}_0 + 1 = \abs{w}_2 }
    \]
\end{exercise}

\textit{Idee}: La fel ca mai sus, dar punem de 4 ori mai multe simboluri pentru \(0\) decât pentru \(2\). La final trebuie să rămână un \(2\) în plus.

\begin{exercise}[exercițiul 10 din examen iunie 2014]
    Construiți un automat pushdown (\pda{}) pentru limbajul
    \[
        \limbaj = \set{ a^n b^n | n \geq 0 } \cup \set{ a^n b^{3n} | n \geq 0 }
    \]
\end{exercise}

\textit{Idee}: Reținem câte \(a\)-uri citim pe stivă. Când dăm de primul \(b\), automatul se bifurcă nedeterminist: pe un caz o să corelăm un \(b\) cu un \(a\), pe altul câte 3 \(b\)-uri cu un \(a\).

\begin{proof}~
    \begin{figure}[H]
        \centering
        \caption*{Construim următorul \pda{}, cu acceptare prin stare finală și stivă vidă:}
        \begin{tikzpicture}
            \node[state, initial] (1) {\(q_1\)};

            \node[state, above right of=1] (2) {\(q_2\)};
            \node[state, accepting, right of=2] (3) {\(q_3\)};

            \node[state, below right of=1] (4) {\(q_4\)};
            \node[state, right of=4] (5) {\(q_5\)};
            \node[state, right of=5] (6) {\(q_6\)};;

            \draw (1) edge[loop above] node[align=center] {
                    \(a, Z_0/AZ_0\) \\
                    \(a, A/AA\)
                } (1)
            (1) edge[below right] node {\(b, A/\lambda\)} (2)
            (1) edge[above right] node {\(b, A/\lambda\)} (4)

            (2) edge[loop above] node[align=center] {
                    \(b, A/\lambda\)
                } (3)
            (2) edge[above] node {\(\lambda, Z_0/Z_0\)} (3)

            (4) edge[above] node[align=center] {
                    \(b, Z_0/Z_0\) \\
                    \(b, A/A\)
                } (5)
            (5) edge[above] node[align=center] {
                    \(b, Z_0/Z_0\) \\
                    \(b, A/A\)
                } (6)
            (6) edge[below, bend left] node {\(b, A/\lambda\)} (4)
            (6) edge[right, bend right] node {\(\lambda, Z_0/Z_0\)} (3);
        \end{tikzpicture}
    \end{figure}

    În starea \(q_1\), automatul citește \(a\)-uri și le pune pe stivă. La primul \(b\) citit, trece nedeterminist în starea \(q_2\) sau \(q_4\).

    În starea \(q_2\) citește \(b\)-uri și le corelează cu \(a\)-uri.

    În stările \(q_4\), \(q_5\) și \(q_6\) avem un ciclu de lungime 3 care corelează câte 3 \(b\)-uri cu un \(a\).
\end{proof}

\begin{exercise}[exercițiul 11 din examen iunie 2017]
    Construiți un automat pushdown (\pda{}) cu acceptare prin stivă vidă pentru limbajul
    \[
        \limbaj = \set{ w | w \in \set{a, b}^*, 2 \abs{w}_a \neq 3 \abs{w}_b + 2 } \cup \set{ aaab, bbba }
    \]
\end{exercise}

\textit{Idee}: Rezolvăm ca mai sus, reținând câte \(a\)-uri / \(b\)-uri am citit pe stivă. La final verificăm să nu se îndeplinească egalitatea.

\begin{exercise}[exercițiul 11 din examen iunie 2018]
    Construiți un automat pushdown (\pda{}) pentru limbajul
    \[
        \limbaj = \set{ w w^R c^i | w \in \set{a, b}^*, i \geq 2 } \cup \set{ abca, ccba, cabc }
    \]
    unde \(w^R\) înseamnă oglinditul lui \(w\), de exemplu \((abcd)^R = dcba\).
\end{exercise}

\textit{Idee}: Reținem pe stivă literele lui \(w\), și alegem nedeterminist care este momentul în care începe \(w^R\).

\begin{proof}~
    \begin{figure}[H]
        \centering
        \caption*{Construim următorul \pda{} nedeterminist:}
        \begin{tikzpicture}
            \node[state, initial] (0) {\(q_0\)};
            \node[state, below right of=0] (dfa) {\(q'\)};
            \coordinate[right of=dfa] (dfa_body);

            \node[state, above right of=0] (1) {\(q_1\)};
            \node[state, right of=1] (2) {\(q_2\)};
            \node[state, right of=2] (3) {\(q_3\)};
            \node[state, accepting, right of=3] (4) {\(q_4\)};

            \draw (0) edge[below left] node {\(\lambda, Z_0/Z_0\)} (dfa)
            (dfa) edge[dotted] (dfa_body)
            (0) edge[above left] node {\(\lambda, Z_0/Z_0\)} (1)
            (1) edge[loop above] node[align=center] {
                    \(a, Z_0/AZ_0\) \\
                    \(a, A/AA\) \\
                    \(a, B/AB\) \\
                    \(b, Z_0/BZ_0\) \\
                    \(b, A/BA\) \\
                    \(b, B/BB\)
                } (1)
            (1) edge[above] node[align=center] {
                    \(\lambda, Z_0/Z_0\) \\
                    \(\lambda, A/A\) \\
                    \(\lambda, B/B\)
                } (2)
            (2) edge[loop above] node[align=center] {
                    \(a, A/\lambda\) \\
                    \(b, B/\lambda\)
                } (2)
            (2) edge[above] node {\(c, Z_0/Z_0\)} (3)
            (3) edge[above] node {\(c, Z_0/Z_0\)} (4)
            (4) edge[loop above] node {\(c, Z_0/Z_0\)} (4);
        \end{tikzpicture}
    \end{figure}

    În \(q_1\) punem pe stivă câte un simbol pentru fiecare literă citită. Alegem nedeterminist care este mijlocul cuvântului, trecând în starea \(q_2\). Aici citim literele și comparăm cu ce avem pe stivă.

    În starea \(q_4\) acceptăm oricâte \(c\)-uri; am verificat deja că avem cel puțin două prin tranzițiile de la \(q_2\) la \(q_3\) și de la \(q_3\) la \(q_4\).

    Începând cu starea \(q'\) avem \dfa{}-ul pentru mulțimea \(\set{ abca, ccba, cabc }\), pe care nu l-am mai scris.
\end{proof}
